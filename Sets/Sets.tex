\documentclass{beamer}

% ========== THEME & PACKAGES ==========
\usetheme{Madrid}
\usepackage{booktabs}
\usepackage{array}
\usepackage{graphicx}
\usepackage{xcolor}

% ========== PRESENTATION INFO ==========
\title[Measure Theory]{Probability and Measure Theory}
\author{Gabriel Asare Okyere (PhD)}
\institute[]{Department of Statistics and Actuarial Science, KNUST, Kumasi, Ghana}
\date{\today}

\setbeamertemplate{navigation symbols}{} % remove navigation icons

% ========== FOOTLINE COLORS ==========
\setbeamercolor{footline-left}{bg=blue!70!black, fg=white}
\setbeamercolor{footline-middle}{bg=blue!60!black, fg=white}
\setbeamercolor{footline-right}{bg=blue!50!black, fg=white}

% ========== CUSTOM FOOTLINE ==========
\setbeamertemplate{footline}{%
	\leavevmode%
	\hbox{%
		% --- Left: Author / Institution ---
		\begin{beamercolorbox}[wd=.33\paperwidth,ht=2.5ex,dp=1ex,center]{footline-left}%
			\textbf{Prof. Okyere (KNUST)}
		\end{beamercolorbox}%
		% --- Middle: Course code ---
		\begin{beamercolorbox}[wd=.33\paperwidth,ht=2.5ex,dp=1ex,center]{footline-middle}%
			\textbf{STAT 751}
		\end{beamercolorbox}%
		% --- Right: Year / Slide number / Logo ---
		\begin{beamercolorbox}[wd=.34\paperwidth,ht=2.5ex,dp=1ex,right]{footline-right}%
			\textbf{2026 \hspace{1em} \insertframenumber{} / \inserttotalframenumber}%
			\hspace{0.5em}%
			\raisebox{-0.5ex}{\includegraphics[height=0.6cm]{knust_logo.jpeg}}%
			\hspace{0.5em}%
		\end{beamercolorbox}%
	}%
	\vskip0pt%
}

% ===== TITLE INFORMATION =====
\title{STAT 751: Measure and Probability Theory}
\author{\Huge{Gabriel Asare Okyere (PhD)}}

\institute{\Large{Department of Statistics and Actuarial Science, KNUST.}}
\date{January 2026}

% ===== DOCUMENT START =====
\begin{document}
	\begin{frame}
		\titlepage
		
	\end{frame}
	
\begin{frame}{Learning objectives}
	\begin{block}{Learning Outcomes}
		\footnotesize % reduce font size to fit content
		At the end of this unit, students are expected to:
		\begin{itemize}
			\item Define and explain fundamental concepts in set theory, including sets, sample spaces, events, and set operations
			\item Classify sets as finite, countably infinite, or uncountable, providing clear examples for each type
			\item Apply set operations, De Morgan's Laws, and the principles of mutually exclusive and exhaustive events in probability scenarios
			\item Construct and analyze sequences of sets---including increasing, decreasing, and disjointized sequences---and describe their convergence behavior
			\item Compare and evaluate key concepts such as partitions versus disjointization and bounded versus unbounded intervals, highlighting their relevance in measure theory and probability
		\end{itemize}
	\end{block}
\end{frame}

	
	
	
	\begin{frame}{Sets}
		\begin{block}{What do you remember about Sets in high school?}
			Consider examples of sets like numbers, letters, or objects you learned.
		\end{block}
		
		\begin{figure}[h!]
			\centering
			\includegraphics[width=0.4\textwidth]{Thinking_4.jpg}
			\caption{Target Outcome}
			\label{fig:thinking}
		\end{figure}
	\end{frame}
	

	
	\begin{frame}{Sets}
		\begin{block}{Definition of Sets}
			A set is a collection of distinct objects, called elements.
		\end{block}
		\vspace{1cm}
		
		\begin{exampleblock}{Examples of sets}
			\begin{itemize}
				\item $A = \{1, 2, 3\}$
				\item $B = \{ \text{names of students in a class}\}$
				\item $C = [0,3]$
				\item $D = \{-3 \leq x \leq 5\}$
			\end{itemize}
		\end{exampleblock}
	\end{frame}
	
	\begin{frame}{Sets}
		\begin{block}{Definition (Sample space)}
			The sample space, written as $\Omega$, is the set of all possible outcomes of an experiment.\\
			Example:
			\begin{itemize}
				\item Tossing a coin: $\Omega = \{H, T\}$
				\item Throwing a die: $\Omega = \{1, 2, 3, 4, 5, 6\}$.
			\end{itemize}
		\end{block}
		\vspace{0.7cm}
		\begin{block}{Event}
			An event is any subset of the sample space.\\
			Example: Let $\Omega = \{1, 2, 3, 4, 5, 6\}$ be the sample space of rolling a fair six-sided die. Let $A$ be the event "the outcome is an even number." Then \\
			$A = \{2, 4, 6\}$.
			
		\end{block}
	\end{frame}
	
	\begin{frame}{Sets}
		\begin{block}{Notations (Set Operations)}
			Let \(\Omega\) denote an abstract space.
			For \(A, B, x \subset \Omega\), we denote\\
			\begin{itemize}
				\item \(A \cup B\) =: \(\{x \in A\) or \( x \in B\}\)
				\item \(A \cap B\) =: \(\{x \in A\) and \( x \in B\}\)
				\item \(A^c\) =: \(\{x \notin A\}\)
				\item \(A\setminus B = A - B = \{x \in A: x \notin B\}\)
				\item  \(A \triangle B\) =: \(\{x \in (A \cap B^c)\) or \( x \in (A^c \cap B)\) but \(x \notin (A \cap B)\}\)
				
			\end{itemize}
		\end{block}
	\end{frame}
	
	\begin{frame}{Sets}
		\begin{block}{Empty set}
			The empty set, denoted by $\emptyset$ or $\{\}$, is the set with no elements.
		\end{block}
		
		\begin{block}{Subset}
			A set $A$ is a subset of $B$, written as $A \subseteq B$, if every element of $A$ is also in $B$.
			
			\begin{itemize}
				\item Proper subset: $A \subset B$ if $A \subseteq B$ and $ A \ne B$. 
			\end{itemize}
		\end{block}
		
		\begin{block}{Power Set}
			The power set of a set $A$, denoted by $\mathcal{P}(A)$, is the set of all subsets of $A$, including the empty set and the set $A$ itself.
			\begin{itemize}
				\item $\mathcal{P}(A) = \{\, B : B \subseteq A \,\}$
				\item If $A$ has $n$ elements, then $\mathcal{P}(A)$ has $2^n$ elements.
			\end{itemize}
		\end{block}
	\end{frame}
	
	\begin{frame}{Sets}
		\begin{block}{Intervals}
			An interval in $\mathbb{R}$ is set of real numbers such that whenever $x$ and $y$ are in the set $x<z<y$, then $z$ is also in the set. 
		\end{block}
		\vspace{0.15cm}
		\begin{block}{Open Interval}
			An open interval is an interval that does not include its endpoints.
			\begin{itemize}
				\item Notation: $(a, b) = \{x \in \mathbb{R}: a<x<b\}$.
			\end{itemize}
		\end{block}
		\vspace{0.3cm}
		\begin{block}{Closed Interval}
			A closed interval is an interval that includes both endpoints.
			\begin{itemize}
				\item Notation: $[a, b] = \{x \in \mathbb{R}: a \leq x \leq b\}$. 
			\end{itemize}
		\end{block}
	\end{frame}
	
	\begin{frame}{Sets}
		\begin{block}{Half-open or Half-closed or Clopen sets}
			A half-open interval includes exactly one endpoint.
			\begin{itemize}
				\item $[a, b) = \{x \in \mathbb{R}: a \leq x < b\}$
				\item $(a, b] = \{x \in \mathbb{R}: a < x \leq b\}$
			\end{itemize}
		\end{block}
		\vspace{0.3cm}
		\begin{block}{Bounded Interval}
			An interval is bounded if it has finite endpoints on both sides.\\
			\begin{itemize}
				\item Example: $[a, b]$, where $a < b$ and both are finite numbers.
				\item Other forms of bounded intervals: $(a, b], [a, b), (a, b)$
			\end{itemize}
		\end{block}
		
		\vspace{0.3cm}
		\begin{block}{Question}
		\Large Can we then say that every interval is bounded?
		\end{block}
	\end{frame}
	
	
	\begin{frame}{Augustus De Morgan}
		
		\begin{center}
			\includegraphics[width=1\textwidth]{DeMorgan1.jpg}  % replace with your image file name
		\end{center}
		
	\end{frame}
	
	
	\begin{frame}{Sets}
		\begin{block}{Unbounded Interval}
			An interval is unbounded if at least one endpoint is infinite.
			\begin{itemize}
				\item Example: $(-\infty, a], (b, +\infty), (-\infty, +\infty)$.
			\end{itemize}
			
	\end{block} 
		\vspace{0.5cm}

		\begin{block}{De Morgan's Laws}
			For any sets $A$ and $B$:
			\begin{itemize}
				\item $(A \cup B)^c = A^c \cap B^c$
				\item $(A \cap B)^c = A^c \cup B^c$
			\end{itemize}
		\end{block}
	\end{frame}
	
	\begin{frame}{Sets}
		\begin{block}{Indexed Families of Sets}
			A collection of sets $\{A_k\}_{k \in I}$ indexed by a set $I$ allows us to define:
		\end{block}
		
		\begin{exampleblock}{Union of an Indexed Family}
			\begin{equation}
				\bigcup_{k \in I} A_k = \{\, x : x \in A_k \text{ for some } k \in I \,\}
			\end{equation}
		\end{exampleblock}
		
		\begin{exampleblock}{Intersection of an Indexed Family}
			\begin{equation}
				\bigcap_{k \in I} A_k = \{\, x : x \in A_k \text{ for all } k \in I \,\}
			\end{equation}
		\end{exampleblock}
	\end{frame}
	
	\begin{frame}{Sets}
		\begin{block}{Sequence of sets}
			A sequence of sets is an ordered collection of sets indexed by the natural numbers. It is written as \\
			$\{A_n\}_{n=1}^{\infty}$ or $A_1, A_2, A_3, \dots$
		\end{block}
		\vspace{1cm}
		\begin{exampleblock}{Note}
			Every sequence of sets is an indexed family.\\
			But not every indexed family is a sequence.
		\end{exampleblock}
	\end{frame}
	
	
	\begin{frame}{Sequences of Sets vs Indexed Families}
		
		\begin{itemize}
			\item \textbf{Example of a sequence of sets (hence an indexed family):}
			
			Let
			\[
			A_1 = (0,1), \quad
			A_2 = (0,2), \quad
			A_3 = (0,3), \ \ldots
			\]
			Then $\{A_n\}_{n\in\mathbb{N}}$ is a sequence of sets, indexed by the natural numbers.
			Since it is indexed, it is also an indexed family.
			
			\item \textbf{Example of an indexed family that is not a sequence:}
			
			Let the index set be $\mathbb{R}$, and define
			\[
			A_t = (-t,t) \quad \text{for } t \in \mathbb{R}.
			\]
			Then $\{A_t\}_{t\in\mathbb{R}}$ is an indexed family of sets, but it is not a sequence,
			because the index set is not $\mathbb{N}$.
		\end{itemize}
		
	\end{frame}
	
	
	\begin{frame}{Sets}
		\begin{block}{Increasing Sequence of sets}
			A sequence of sets $A_1, A_2, A_3, \dots$ is increasing if each set is contained in the next one:\\ $A_1 \subset A_2 \subset A_3 \subset \dots $.
			
		\end{block} \vspace{1cm}
		
		\begin{block}{Decreasing Sequence of sets}
			A sequence of sets $A_1, A_2, A_3, \dots$ is decreasing if each set contains the next one:\\ $A_1 \supset A_2 \supset A_3 \supset \dots$
		\end{block}
	\end{frame}
	
	
	\begin{frame}{Monotone Convergence of Sets}
		\begin{block}{Increasing case}
			A sequence of sets $(A_n)_{n=1}^{\infty}$ converges increasingly to a set $A$ if: \\
			\begin{itemize}
				\item the sets are increasing (as shown above)
				\item the limit set $A$ is the union of all sets: i.e. $A = \displaystyle \bigcup_{n=1}^{\infty} A_n$
				\item We write: $A_n \uparrow A$.
			\end{itemize}
		\end{block}
	\end{frame}
	
	\begin{frame}{Monotone Convergence of Sets}
		\begin{block}{Decreasing case}
			A sequence of sets $(A_n)_{n=1}^{\infty}$ converges decreasingly to a set $A$ if: \\
			\begin{itemize}
				\item the sets are decreasing (as shown earlier)
				\item the limit set $A$ is the intersection of all sets: i.e. $A = \displaystyle \bigcap_{n=1}^{\infty} A_n$
				\item We write: $A_n \downarrow A$.
			\end{itemize}
		\end{block}
	\end{frame}
	
	\begin{frame}{Sets}
		\begin{block}{Partition of a Set}
			The partition of a set $A$ is a collection of disjoint subsets $\{A_i\}$ such that:
			\begin{itemize}
				\item $A_i \cap A_j = \emptyset$ for $i \neq j$
				\item $\bigcup_i A_i = A$
			\end{itemize}
		\end{block}
	\end{frame}
	
	\begin{frame}{Disjointization}
		\begin{block}{Definition}
			Given a sequence of sets $A_1, A_2, A_3, \dots,$ disjointization creates a sequence of disjoint sets $B_1, B_2, B_3 \dots$, such that:\\
			\begin{itemize}
				\item $B_i \cap B_j = \emptyset$ for $i \neq j$ (disjoint).
				\item The union is preserved: \\
				$\displaystyle \bigcup_{n=1}^{\infty} A_n = \bigcup_{n=1}^{\infty} B_n$.
			\end{itemize}
			
		\end{block}
	\end{frame}
	
	\begin{frame}{Disjointization of an Increasing Sequence of Sets}
		\begin{block}{Definition}
			\begin{itemize}
				\item Let $(A_n)$ be an increasing sequence of sets, meaning \\ $A_1 \subset A_2 \subset A_3 \subset \dots$.\\
				
				\item The disjointization of $(A_n)$ is the sequence of disjoint sets $(B_n)$ defined by \\
				$B_1 = A_1$, \quad $B_n = A_n \setminus A_{n-1}$ \quad $(n \ge 2)$
				
				\item These sets satisfy:\\ $B_i \cap B_j = \emptyset$ for $i \neq j$ (they are pairwise disjoint).
				
				\item Their union equals the union of the original sequence:\\
				$\displaystyle \bigcup_{n=1}^{\infty} A_n = \bigcup_{n=1}^{\infty} B_n$
			\end{itemize} 
		\end{block}
	\end{frame}
	
	
	\begin{frame}{Examples of Disjointization}
		
		\textbf{Example 1: Overlapping intervals on the real line}
		
		Let
		\[
		A_1 = (0,2), \quad
		A_2 = (1,3), \quad
		A_3 = (2,4).
		\]
		
		Disjointization gives
		\[
		B_1 = (0,2),
		\]
		\[
		B_2 = (1,3)\setminus(0,2) = [2,3),
		\]
		\[
		B_3 = (2,4)\setminus\big((0,2)\cup(1,3)\big) = [3,4).
		\]
		
		Then the sets \(B_1, B_2, B_3\) are disjoint and
		\[
		\bigcup_{n=1}^3 B_n = (0,4).
		\]
		
		\vspace{0.3cm}
		
		\textbf{Example 2: Repeated events in probability}
		
		Let \((\Omega,\mathcal{F},P)\) be a probability space, and define
		\[
		A_1 = \{\text{first toss is a head}\},
		\]
		
	\end{frame}
	
	\begin{frame}{Examples of disjointization}
		\[
		A_2 = \{\text{at least one head in two tosses}\},
		\]
		\[
		A_3 = \{\text{at least one head in three tosses}\}.
		\]
		
		Disjointization gives
		\[
		B_1 = A_1,
		\]
		\[
		B_2 = A_2 \setminus A_1 = \{\text{first tail, second head}\},
		\]
		\[
		B_3 = A_3 \setminus (A_1 \cup A_2)
		= \{\text{first two tails, third head}\}.
		\]
		
		Each \(B_n\) represents the event that the first head occurs exactly at time \(n\).
		
		\vspace{0.3cm}
		
		\textbf{Example 3: Measurable sets in integration}
		
		Let \(f:\Omega \to [0,\infty)\) be a measurable function and define
		\[
		A_n = \{\omega \in \Omega : f(\omega) > n\}, \quad n \in \mathbb{N}.
		\]
		
		The disjointization is
		\[
		B_1 = A_1,
		\]
		
		\end{frame}
		
		\begin{frame}{Examples of disjointization}
		\[
		B_n = A_n \setminus A_{n-1}
		= \{\omega : n < f(\omega) \le n+1\}.
		\]
		
		The sets \(B_n\) are disjoint and partition the values of \(f\) into non-overlapping levels.
		
	\end{frame}
	
	
	
	\begin{frame}{Discussion}
		Compare and contrast Partition and Disjointization.
	\end{frame}
	
	
	\begin{frame}{Sets}
		\begin{block}{Countability}
			\begin{itemize}
				\item A set is finite if it has a limited number of elements. Eg. $A = \{2, 4, 6,8,10\}$
			\end{itemize}
		\end{block}
	\end{frame}
	
	
	\begin{frame}{Sets}
		
		\begin{block}{Definition of Infinite Sample Space}
			A \textbf{sample space} $\Omega$ in probability is called \textbf{infinite} if it contains infinitely many outcomes. That is,
			\[
			|\Omega| = \infty.
			\]
		\end{block}
		
		\begin{itemize}
			\item \textbf{Countably infinite:} the outcomes can be listed as $\omega_1, \omega_2, \omega_3, \dots$ (e.g., tossing a coin until the first head)
			\item \textbf{Uncountably infinite:} the outcomes cannot be listed in a sequence (e.g., choosing a real number in the interval $[0,1]$)
		\end{itemize}
		
		\textbf{Plain explanation:} An infinite sample space means there are endlessly many possible outcomes, either in a way we can count one by one (countable) or not (uncountable).
		
	\end{frame}
	
	
	\begin{frame}{Sets}
		\begin{block}{Mutually Exclusive Events}
			Two events $A$ and $B$ are mutually exclusive if they cannot occur at the same time.
			\begin{itemize}
				\item Formally: $ A \cap B = \emptyset$.
			\end{itemize}
		\end{block}
		
		\vspace{0.5cm}
		
		\begin{block}{Exhaustive events}
			A collection of events $\{A_i\}$ is exhaustive if at least one of them must occur.
			\begin{itemize}
				\item Formally: $\displaystyle \bigcup_{i} A_i = \Omega$
			\end{itemize}
		\end{block}
	\end{frame}
	
	
	\begin{frame}{Assignment}
		With the aid of examples, discuss the similarities and differences between finite, countably infinite, and uncountable sets. 
	\end{frame}
	
\begin{frame}{References}
	\begin{thebibliography}{99}
		
		\bibitem{grigoryan}
		A. Grigoryan,
		\textit{Measure Theory and Probability},
		University of Bielefeld, Lecture Notes.
		
		\bibitem{athreya}
		K. B. Athreya and S. N. Lahiri,
		\textit{Measure and Probability Theory},
		Springer, 2006.
		
		\bibitem{billingsley}
		P. Billingsley,
		\textit{Probability and Measure},
		3rd ed., Wiley, 1995.
		
		\bibitem{youtube}
		A Probability Space,
		\textit{Measure Theory and Probability},
		YouTube lecture series, 2023.\\
		Link: \url{https://youtu.be/swa1VRYms3Q}
	\end{thebibliography}

\end{frame}

	\end{document}
