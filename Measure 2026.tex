\documentclass{beamer}

% ========== THEME & PACKAGES ==========
\usetheme{Madrid}
\usepackage{booktabs}
\usepackage{array}
\usepackage{graphicx}
\usepackage{xcolor}

% ========== PRESENTATION INFO ==========
\title[Measure Theory]{Measure and Probability Theory}
\author{Gabriel Asare Okyere (PhD)}
\institute[]{Department of Statistics and Actuarial Science, KNUST, Kumasi, Ghana}
\date{\today}

\setbeamertemplate{navigation symbols}{} % remove navigation icons

% ========== FOOTLINE COLORS ==========
\setbeamercolor{footline-left}{bg=blue!70!black, fg=white}
\setbeamercolor{footline-middle}{bg=blue!60!black, fg=white}
\setbeamercolor{footline-right}{bg=blue!50!black, fg=white}

% ========== CUSTOM FOOTLINE ==========
\setbeamertemplate{footline}{%
	\leavevmode%
	\hbox{%
		% --- Left: Author / Institution ---
		\begin{beamercolorbox}[wd=.33\paperwidth,ht=2.5ex,dp=1ex,center]{footline-left}%
			\textbf{Prof. Okyere (KNUST)}
		\end{beamercolorbox}%
		% --- Middle: Course code ---
		\begin{beamercolorbox}[wd=.33\paperwidth,ht=2.5ex,dp=1ex,center]{footline-middle}%
			\textbf{STAT ***}
		\end{beamercolorbox}%
		% --- Right: Year / Slide number / Logo ---
		\begin{beamercolorbox}[wd=.34\paperwidth,ht=2.5ex,dp=1ex,right]{footline-right}%
			\textbf{2026 \hspace{1em} \insertframenumber{} / \inserttotalframenumber}%
			\hspace{0.5em}%
			\raisebox{-0.5ex}{\includegraphics[height=0.6cm]{knust_logo.jpeg}}%
			\hspace{0.5em}%
		\end{beamercolorbox}%
	}%
	\vskip0pt%
}

% ===== TITLE INFORMATION =====
\title{STAT ***: Measure and Probability Theory}
\author{\Huge{Gabriel Asare Okyere (PhD)}}

\institute{\Large{Department of Statistics and Actuarial Science, KNUST.}}
\date{January 2026}

% ===== DOCUMENT START =====
\begin{document}
	\begin{frame}
		\titlepage
		
	\end{frame}
	
	\begin{frame}{Learning objectives}
		\begin{block}{}
			At the end of this unit, students are expected to:
			\begin{itemize}
				\item Define fundamental concepts of sets, algebra, and measure theory.
				\item Explain the properties and relationships between measurable sets, spaces, and measures.
				\item Construct measurable spaces and apply measure theory in advanced scenarios.
				\item Understand set sequences, outer measures, and measures.
			\end{itemize}
		\end{block}
		
	\end{frame}
	
	
	
	\begin{frame}{Sets}
		\begin{block}{What do you remember about Sets in high school?}
			Think about examples like numbers, letters, or objects you learned.
		\end{block}
	\end{frame}
	
	
	
	\begin{frame}	
	\begin{figure}[h!]
			\centering
			\includegraphics[width=0.45\textwidth]{Thinking_4.jpg}
			\caption{Target Outcome}
			\label{fig:thinking}
	\end{figure}
	\end{frame}



	
	\begin{frame}{Sets}
		\begin{block}{Definition (Sets)}
			A set is a collection of distinct objects, called elements.
		\end{block}
		\vspace{1cm}
		
		\begin{exampleblock}{Examples of sets}
			\begin{itemize}
				\item $A = \{1, 2, 3\}$
				\item $B = \{ \text{names of students in a class}\}$
			\end{itemize}
		\end{exampleblock}
	\end{frame}
	
	\begin{frame}{Sets}
		\begin{block}{Definition (Sample space)}
			The sample space, written as $\Omega$, is the set of all possible outcomes of an experiment.\\
			Example:
			\begin{itemize}
				\item Tossing a coin: $\Omega = \{H, T\}$
				\item Throwing a die: $\Omega = \{1, 2, 3, 4, 5, 6\}$.
			\end{itemize}
		\end{block}
		\vspace{0.7cm}
		\begin{block}{Event}
			An event is any subset of the sample space.\\
			Example: Let $\Omega = \{1, 2, 3, 4, 5, 6\}$ be the sample space of rolling a fair six-sided die. Let $A$ be the event "the outcome is an even number." Then \\
			$A = \{2, 4, 6\}$.
			
		\end{block}
	\end{frame}
	
	\begin{frame}{Sets}
		\begin{block}{Definitions of Notations (Set Operations)}
			Let \(\Omega\) denote an abstract space.
			For \(A, B, x \subset \Omega\), we denote\\
			\begin{itemize}
				\item \(A \cup B\) =: \(\{x \in A\) or \( x \in B\}\)
				\item \(A \cap B\) =: \(\{x \in A\) and \( x \in B\}\)
				\item \(A^c\) =: \(\{x \notin A\}\)
				\item \(A\setminus B = A - B = \{x \in A: x \notin B\}\)
				\item  \(A \triangle B\) =: \(\{x \in (A \cap B^c)\) or \( x \in (A^c \cap B)\) but \(x \notin (A \cap B)\}\)
				
			\end{itemize}
		\end{block}
	\end{frame}
	
	\begin{frame}{Sets}
		\begin{block}{Empty set}
			The empty set, denoted by $\emptyset$ or $\{\}$, is the set with no elements.
		\end{block}
		
		\begin{block}{Subset}
			A set $A$ is a subset of $B$, written as $A \subseteq B$, if every element of $A$ is also in $B$.
			
			\begin{itemize}
				\item Proper subset: $A \subset B$ if $A \subseteq B$ and $ A \ne B$. 
			\end{itemize}
		\end{block}
		
		\begin{block}{Power Set}
			The power set of a set $A$, denoted by $\mathcal{P}(A)$, is the set of all subsets of $A$, including the empty set and the set $A$ itself.
			\begin{itemize}
				\item $\mathcal{P}(A) = \{\, B : B \subseteq A \,\}$
				\item If $A$ has $n$ elements, then $\mathcal{P}(A)$ has $2^n$ elements.
			\end{itemize}
		\end{block}
	\end{frame}
	
	\begin{frame}{Sets}
		\begin{block}{Intervals}
			An interval in $\mathbb{R}$ is set of real numbers such that whenever $x$ and $y$ are in the set $x<z<y$, then $z$ is also in the set. 
		\end{block}
		\vspace{0.15cm}
		\begin{block}{Open Interval}
			An open interval is an interval that does not include its endpoints.
			\begin{itemize}
				\item Notation: $(a, b) = \{x \in \mathbb{R}: a<x<b\}$.
			\end{itemize}
		\end{block}
		\vspace{0.3cm}
		\begin{block}{Closed Interval}
			A closed interval is an interval that includes both endpoints.
			\begin{itemize}
				\item Notation: $[a, b] = \{x \in \mathbb{R}: a \leq x \leq b\}$. 
			\end{itemize}
		\end{block}
	\end{frame}
	
	\begin{frame}{Sets}
		\begin{block}{Half-open or Half-closed or Clopen sets}
			A half-open interval includes exactly one endpoint.
			\begin{itemize}
				\item $[a, b) = \{x \in \mathbb{R}: a \leq x < b\}$
				\item $(a, b] = \{x \in \mathbb{R}: a < x \leq b\}$
			\end{itemize}
		\end{block}
		\vspace{0.3cm}
		\begin{block}{Bounded Interval}
			An interval is bounded if it has finite endpoints on both sides.\\
			\begin{itemize}
				\item Example: $[a, b]$, where $a < b$ and both are finite numbers.
				\item Other forms of bounded intervals: $(a, b], [a, b), (a, b)$
			\end{itemize}
		\end{block}
		
		\vspace{0.3cm}
		\begin{block}{Unbounded Interval}
			An interval is unbounded if at least one endpoint is infinite.
			\begin{itemize}
				\item Example: $(-\infty, a], (b, +\infty), (-\infty, +\infty)$.
			\end{itemize}
		\end{block}
	\end{frame}
	
	
	\begin{frame}{Sets}
		\begin{block}{De Morgan's Laws}
			For any sets $A$ and $B$:
			\begin{itemize}
				\item $(A \cup B)^c = A^c \cap B^c$
				\item $(A \cap B)^c = A^c \cup B^c$
			\end{itemize}
		\end{block}
	\end{frame}
	
	\begin{frame}{Sets}
		\begin{block}{Indexed Families of Sets}
			A collection of sets $\{A_k\}_{k \in I}$ indexed by a set $I$ allows us to define:
		\end{block}
		
		\begin{exampleblock}{Union of an Indexed Family}
			\begin{equation}
				\bigcup_{k \in I} A_k = \{\, x : x \in A_k \text{ for some } k \in I \,\}
			\end{equation}
		\end{exampleblock}
		
		\begin{exampleblock}{Intersection of an Indexed Family}
			\begin{equation}
				\bigcap_{k \in I} A_k = \{\, x : x \in A_k \text{ for all } k \in I \,\}
			\end{equation}
		\end{exampleblock}
	\end{frame}
	
	\begin{frame}{Sets}
		\begin{block}{Sequence of sets}
			A sequence of sets is an ordered collection of sets indexed by the natural numbers. It is written as \\
			$\{A_n\}_{n=1}^{\infty}$ or $A_1, A_2, A_3, \dots$
		\end{block}
		\vspace{1cm}
		\begin{block}{Remark}
			Every sequence of sets is an indexed family.\\
			But not every indexed family is a sequence.
		\end{block}
	\end{frame}
	
	\begin{frame}{Sets}
		\begin{block}{Increasing Sequence of sets}
			A sequence of sets $A_1, A_2, A_3, \dots$ is increasing if each set is contained in the next one:\\ $A_1 \subset A_2 \subset A_3 \subset \dots $.
			
		\end{block} \vspace{1cm}
		
		\begin{block}{Decreasing Sequence of sets}
			A sequence of sets $A_1, A_2, A_3, \dots$ is decreasing if each set contains the next one:\\ $A_1 \supset A_2 \supset A_3 \supset \dots$
		\end{block}
	\end{frame}
	
	
	\begin{frame}{Monotone Convergence of Sets}
		\begin{block}{Increasing case}
			A sequence of sets $(A_n)_{n=1}^{\infty}$ converges increasingly to a set $A$ if: \\
			\begin{itemize}
				\item the sets are increasing (as shown above)
				\item the limit set $A$ is the union of all sets: i.e. $A = \displaystyle \bigcup_{n=1}^{\infty} A_n$
				\item We write: $A_n \uparrow A$.
			\end{itemize}
		\end{block}
	\end{frame}
	
	\begin{frame}{Monotone Convergence of Sets}
		\begin{block}{Decreasing case}
			A sequence of sets $(A_n)_{n=1}^{\infty}$ converges decreasingly to a set $A$ if: \\
			\begin{itemize}
				\item the sets are decreasing (as shown earlier)
				\item the limit set $A$ is the intersection of all sets: i.e. $A = \displaystyle \bigcap_{n=1}^{\infty} A_n$
				\item We write: $A_n \downarrow A$.
			\end{itemize}
		\end{block}
	\end{frame}
	
	\begin{frame}{Sets}
		\begin{block}{Partition of a Set}
			A partition of a set $A$ is a collection of disjoint subsets $\{A_i\}$ such that:
			\begin{itemize}
				\item $A_i \cap A_j = \emptyset$ for $i \neq j$
				\item $\bigcup_i A_i = A$
			\end{itemize}
		\end{block}
	\end{frame}
	
	\begin{frame}{Disjointization}
		\begin{block}{Definition}
			Given a sequence of sets $A_1, A_2, A_3, \dots,$ disjointization creates a sequence of disjoint sets $B_1, B_2, B_3 \dots$, such that:\\
			\begin{itemize}
				\item $B_i \cap B_j = \emptyset$ for $i \neq j$ (disjoint).
				\item The union is preserved: \\
				$\displaystyle \bigcup_{n=1}^{\infty} A_n = \bigcup_{n=1}^{\infty} B_n$.
			\end{itemize}
			
		\end{block}
	\end{frame}
	
	\begin{frame}{Disjointization of an Increasing Sequence of Sets}
		\begin{block}{Definition}
			\begin{itemize}
				\item Let $(A_n)$ be an increasing sequence of sets, meaning \\ $A_1 \subset A_2 \subset A_3 \subset \dots$.\\
				
				\item The disjointization of $(A_n)$ is the sequence of disjoint sets $(B_n)$ defined by \\
				$B_1 = A_1$, \quad $B_n = A_n \setminus A_{n-1}$ \quad $(n \ge 2)$
				
				\item These sets satisfy:\\ $B_i \cap B_j = \emptyset$ for $i \neq j$ (they are pairwise disjoint).
				
				\item Their union equals the union of the original sequence:\\
				$\displaystyle \bigcup_{n=1}^{\infty} A_n = \bigcup_{n=1}^{\infty} B_n$
			\end{itemize} 
		\end{block}
	\end{frame}
	
	
	\begin{frame}{Discussion}
		Compare and contrast Partition and Disjointization.
	\end{frame}
	
	
	\begin{frame}{Countability}
		\begin{block}{Countability}
			\begin{itemize}
				\item A set is finite if it has a limited number of elements. Eg. $A = \{2, 4, 6,8,10\}$
				\item Countably infinite: elements can be listed as $a_1, a_2, a_3, \dots$ (e.g., $\mathbb{N}, \mathbb{Q}$)
				\item Uncountable: cannot be listed in a sequence (e.g., $\mathbb{R}$)
			\end{itemize}
		\end{block}
	\end{frame}
	
	\begin{frame}{Sets}
		\begin{block}{Mutually Exclusive Events}
			Two events $A$ and $B$ are mutually exclusive if they cannot occur at the same time.
			\begin{itemize}
				\item Formally: $ A \cap B = \emptyset$.
			\end{itemize}
		\end{block}
		
		\begin{block}{Exhaustive events}
			A collection of events $\{A_i\}$ is exhaustive if at least one of them must occur.
			\begin{itemize}
				\item Formally: $\displaystyle \bigcup_{i} A_i = \Omega$
			\end{itemize}
		\end{block}
	\end{frame}
	
	\begin{frame}{Towards Algebra -- The Idea}
		So far, we have worked with individual sets such as intervals and simple events. \\
		To define length, probability, or measure, we need to work with collections of sets, not just single sets.\\
		We want a collection of sets with the following properties:
		\begin{itemize}
			\item the empty set should be included
			\item complements should remain inside the collection
			\item unions of sets should stay inside the collection
		\end{itemize}
		
		However, requiring all these properties at once is often too strong at the beginning.\\
		Instead, we start with very simple building blocks, such as intervals, and gradually add structure.
	\end{frame}
	
	\begin{frame}{Semi-ring of Sets}
		\begin{block}{Definition (Semi-ring)}
			A family $\mathcal{S}$ of subsets of $\Omega$ is called a semi-ring if:
			\begin{itemize}
				\item $\emptyset \in \mathcal{S}$
				\item If $A, B \in \mathcal{S}$, then $A \cap B \in \mathcal{S}$
				\item If $A, B \in \mathcal{S}$, then $A \setminus B$ is a disjoint union of a finite family of sets from $\mathcal{S}$
			\end{itemize}
		\end{block}
	\end{frame}
	
	
	\begin{frame}{Semi-Ring}
		\begin{block}{Example}
			Consider the set of intervals in $\mathbb{R}$:
			\begin{equation*}
				\mathcal{S} = \{ [a, b) \mid a < b, \ a, b \in \mathbb{R} \}.
			\end{equation*}
		\end{block}    
	\end{frame}
	
	\begin{frame}{Semi-ring}
		\begin{exampleblock}{Proof that $\mathcal{S}$ is a semi-ring.}
			\begin{itemize}
				\item Take $[a, a) = \emptyset$, so $\emptyset \in \mathcal{S}$.
				\item The intersection $[a, b) \cap [c, d) = [\max(a,c), \min(b,d)) \in \mathcal{S}$.
				\item Suppose that $[a, b) \subseteq [c, d)$. The difference $[a, b) \setminus [c, d)$ is a finite union of disjoint sets from $\mathcal{S}$. \\
				That is, $[a, b) \setminus [c, d) = [c, a) \cup [b, d)$.
			\end{itemize}
		\end{exampleblock}
		
		\begin{exampleblock}{Other examples of semi-ring}
			\begin{itemize}
				\item The family of all intervals in $\mathbb{R}$ is a semi-ring.
			\end{itemize}
		\end{exampleblock}
	\end{frame}
	
	\begin{frame}{Ring}
		\begin{block}{Definition}
			A family $S$ of subsets of $\Omega$ is called a ring if
			\begin{itemize}
				\item $S$ contains $\emptyset$
				\item $A, B \in S \implies A \cup B \in S,$ and $A \setminus B \in S$.
			\end{itemize} \vspace{0.25cm}
			It follows also that the intersection $A \cap B$ belongs to $S$ because \\ $A \cap B = B \setminus (B \setminus A)$ is also in $S$. \\
			Also, it follows that a ring is a semi-ring.
		\end{block}
	\end{frame}
	
	\begin{frame}{Ring}
		\begin{block}{Example}
			Let $\Omega = \{1, 2, 3, 4,5, \dots\}$ and let $\mathcal{R} = \{ \text{All finite subsets of $\Omega$}\}$.
		\end{block}
		\begin{exampleblock}{Proof that $\mathcal{R}$ is a ring}
			\begin{itemize}
				\item $\emptyset \subseteq \Omega$ and is finite, therefore $\emptyset \in \mathcal{R}$
				\item $\{1, 3\} \cup \{2, 4, 5\} = \{1, 2, 3, 4, 5\} \in \mathcal{R}$.
				\item $\{1, 2, 3\} \setminus \{2\} = \{1, 3\} \in \mathcal{R}$.
			\end{itemize}
		\end{exampleblock}
		\begin{alertblock}{Other examples of a ring}
			\begin{itemize}
				\item $\mathcal{R} = \{\text{all finite subsets of } \mathbb{N}\} \text{ is a ring.}$
				\item $\mathcal{R} = \{[a, b) \subseteq \mathbb{R} | a, b \in \mathbb{Q}, a \leq b\}$. That is, all intervals of this form on the real line with rational endpoints.
			\end{itemize}
			
		\end{alertblock}
	\end{frame}
	
	
	\begin{frame}{$\sigma$-ring}
		\begin{block}{Definition}
			A ring $S$ is called a $\sigma$-ring if the union of any countable family $\{A_k\}_{k=1}^{\infty}$ of sets from $S$ is also in $S$. That is:
			\begin{itemize}
				\item Closure under unions: If $A_1, A_2, A_3, \dots \in \mathcal{R}$, then \\
				$\displaystyle \bigcup_{n=1}^{\infty} A_n \in \mathcal{R}$. 
				\item Closure under difference: If $A, B \in \mathcal{R}$, then $A \setminus B \in \mathcal{R}$.
			\end{itemize}
		\end{block}
		
		\begin{exampleblock}{}
			\begin{itemize}
				\item It follows that the intersection $A = \bigcap_{k} A_k$ is also in $S$.
				\item Let $B$ be any of the sets $A_k$ so that $B \supset A$, then \\
				$A = B \setminus (B \setminus A) = B \setminus \left (\bigcup_{k} (B \setminus A_k) \right ) \in S$.
			\end{itemize} 
		\end{exampleblock}
	\end{frame}
	
	\begin{frame}{Algebra}
		\begin{block}{Definition (Algebra)}
			Let $\Omega$ be a non-empty set.\\
			A collection of subsets of $\Omega$ is called an algebra $\mathcal{F}$, if it satisfies the following conditions:
			\begin{enumerate}
				\item $\Omega \in \mathcal{F}$
				\item $A \in \mathcal{F} \implies A^c \in \mathcal{F}$
				\item If $A_1, A_2, \dots, A_n \in \mathcal{F}$, then  $\bigcup_{i=1}^{n} A_i \in \mathcal{F}$
			\end{enumerate}
		\end{block}
		\vspace{1cm}
		
		\begin{exampleblock}{Note}
			A ring containing $\Omega$ is called an algebra.
		\end{exampleblock}
	\end{frame}
	
	\begin{frame}{Algebra}
		\begin{exampleblock}{Example}
			Given $\Omega = \{1, 2, 3\}$. Construct the algebra $\mathcal{F}$.
		\end{exampleblock}
		
	\end{frame}
	
	\begin{frame}{Algebra}
		\begin{exampleblock}{Solution}
			\begin{itemize}
				\item $\Omega = \{1, 2, 3\}$
				\item $\mathcal{F} = \{\emptyset, \Omega, \{2\}, \{1,3\}\}$
			\end{itemize}
			Clearly, $\mathcal{F}$ satisfies all three properties of an algebra. 
		\end{exampleblock} \vspace{1cm}
		
		\begin{alertblock}{Note}
			The power set of $\Omega$, denoted by $\mathcal{P}(\Omega)$ or $2^\Omega$ is an algebra.
		\end{alertblock}
	\end{frame}
	
	
	\begin{frame}{$\sigma$-Algebra}
		\begin{block}{Definition}
			A collection of sets $\mathcal{F}$ is called a $\sigma$-algebra if it is an algebra and satisfies the property:
			\begin{itemize}
				\item If $A_n \in \mathcal{F}$ for $n \ge 1$, then $\bigcup_{n \ge 1} A_n \in \mathcal{F}$.
			\end{itemize}
			That is, a $\sigma$-algebra is an algebra and is closed under countable unions.
		\end{block}
		
		\begin{block}{}
			\begin{itemize}
				\item[(i)] $\Omega \in \mathcal{F}$,
				\item[(ii)] $A \in \mathcal{F} \implies A^c \in \mathcal{F}$,
				\item[(iii)] $A_1, A_2, \dots \in \mathcal{F} \implies \bigcup_{j=1}^\infty A_j \in \mathcal{F}$
			\end{itemize}
		\end{block}
	\end{frame}
	
	
	\begin{frame}{Some general, simple examples of $\sigma$-algebras}
		\begin{block}{Examples}
			\begin{enumerate}
				\item $\mathcal{F} = \{\emptyset, \Omega\}$ --- \textit{trivial $\sigma$-algebra}
				\item $\mathcal{F} = \{\text{all subsets of } \Omega\}$ --- The largest $\sigma$-algebra
				\item Let $\mathcal{A} = \{A\} \subset \Omega$, then \\ 
				$\sigma(\mathcal{A}) = \{\emptyset, A, A^c, \Omega\}$
			\end{enumerate}
		\end{block}
	\end{frame}
	
	\begin{frame}{Example}
		\begin{block}{}
			Given $\Omega = \{HH, TH, HT, TT\}$. Construct the following on $\Omega$.
			\begin{itemize}
				\item $\sigma(\{HH, TT\})$
				\item $\sigma(\{TH\})$
				\item $\sigma(\{\emptyset\})$
				\item $\sigma(\{TH\}, \{HT\})$
			\end{itemize}
		\end{block}
		
		\vspace{1cm}
		
		\begin{exampleblock}{Note}
			A $\sigma$-ring on $\Omega$ containing $\Omega$ is called a $\sigma$-algebra.
		\end{exampleblock}
	\end{frame}
	
	
	
	\begin{frame}{MEASURE}
		WHAT IS MEASURE AND WHY DO YOU THINK WE NEED MEASURE IN THIS CONTEXT?
	\end{frame}
	
	
	\begin{frame}{Why measure?}
		
		In mathematics, we often ask questions like:
		\begin{itemize}
			\item how long is this set?
			\item how large is this region?
			\item what is the chance that an event happens?
		\end{itemize}
		
		To answer these questions, we need a systematic way to assign numbers to sets.
		
		This is the motivation for measure theory.
		
	\end{frame}
	
	
	\begin{frame}{The problem with ``size''}
		
		For simple sets, ``size'' is easy.
		
		\vspace{0.3cm}
		
		Examples:
		\begin{itemize}
			\item the length of an interval $(a,b)$ is $b-a$
			\item the number of elements in a finite set is easy to count
		\end{itemize}
		
		\vspace{0.3cm}
		
		But it is not simple for many sets.
		
		\vspace{0.3cm}
		
		Examples:
		\begin{itemize}
			\item unions of many intervals
			\item infinite sets
			\item complicated events in probability
		\end{itemize}
		
		\vspace{0.3cm}
		
		We need a more general idea of size.
		
	\end{frame}
	
	
	\begin{frame}{Making it slightly more complicated}
		
		Now take two intervals that do not overlap.
		
		\vspace{0.3cm}
		
		\[
		A = (1,2), \qquad B = (3,5)
		\]
		
		Their lengths are:
		\begin{itemize}
			\item $\text{length}(A) = 1$
			\item $\text{length}(B) = 2$
		\end{itemize}
		
		\vspace{0.3cm}
		
		The total length of $A \cup B$ is
		\[
		1 + 2 = 3
		\]
		
		So far, everything works nicely.
		
	\end{frame}
	
	
	\begin{frame}{Limitations of basic ideas}
		
		Length works well for single intervals,  
		but what about:
		\begin{itemize}
			\item a countable union of intervals?
			\item sets with infinitely many pieces?
			\item random events formed from many outcomes?
		\end{itemize}
		
		Basic formulas are not enough.
		
		\vspace{0.3cm}
		
		So we need a rule that:
		\begin{itemize}
			\item works for simple sets
			\item extends to complicated sets
			\item behaves consistently
		\end{itemize}
		
	\end{frame}
	
	
	\begin{frame}{From size to measure}
		
		A measure is a mathematical tool that:
		\begin{itemize}
			\item assigns a size to sets
			\item works for very general sets
			\item satisfies the natural properties we expect
		\end{itemize}
		
		\vspace{0.3cm}
		
		Length, area, volume, and probability are all special cases of measures.
		
		\vspace{0.3cm}
		
		This motivates the formal definition of a measure.
		
	\end{frame}
	
	\begin{frame}{Measure}
		\begin{block}{Definition of a Measure}
			Let $\Omega$ be a set and $\mathcal{F}$ a collection of subsets of $\Omega$ (called a \(\sigma\)-algebra).  
			
			A function \(\mu: \mathcal{F} \to [0, \infty]\) is called a \textbf{measure} if it satisfies:
			
			\vspace{0.15cm}
			
			\begin{itemize}
				\item \(\mu(\emptyset) = 0\)
				\item \textbf{Countable additivity:} for any countable collection \(\{A_1, A_2, A_3, \dots\}\) of disjoint sets in \(\mathcal{F}\),
				\[
				\mu\Big(\bigcup_i A_i\Big) = \sum_i \mu(A_i)
				\]
			\end{itemize}
		\end{block}
		\end{frame}
		
		\begin{frame}{Measure}
		\begin{block}{Remarks}
			\begin{itemize}
				\item $\mu$ assigns a non-negative extended real number (can be $\infty$) to each measurable set.
				\item The $\sigma$-algebra $\mathcal{F}$ ensures that unions, intersections, and complements of sets are measurable.
				\item Countable additivity is stronger than finite additivity—it works for infinitely many disjoint sets.
			\end{itemize}
		\end{block}
		\end{frame}
		
	\begin{frame}{Probability Measure}
		
		A probability measure $P$ is a measure on a $\sigma$-algebra $\mathcal{F}$ of $\Omega$ such that:
		
		\vspace{0.3cm}
		
		\begin{itemize}
			\item $P(A) \ge 0$ for all $A \in \mathcal{F}$
			\item $P(\Omega) = 1$
			\item For any countable collection $\{A_1, A_2, \dots\}$ of disjoint events in $\mathcal{F}$,
			\[
			P\Big(\bigcup_i A_i\Big) = \sum_i P(A_i)
			\]
		\end{itemize}
		
	\end{frame}
	
	
	
	\begin{frame}{Measure Space}
		
		A measure space is an ordered triple $(\Omega, \mathcal{F}, \mu)$ where:
		
		\vspace{0.3cm}
		
		\begin{itemize}
			\item $\Omega$ is a set, called the sample space.
			\item $\mathcal{F}$ is a $\sigma$-algebra of subsets of $\Omega$, i.e., a collection of subsets that satisfies:
			\begin{itemize}
				\item $\Omega \in \mathcal{F}$
				\item if $A \in \mathcal{F}$, then $A^c \in \mathcal{F}$ (complement is measurable)
				\item if $A_1, A_2, \dots \in \mathcal{F}$, then $\bigcup_{i=1}^{\infty} A_i \in \mathcal{F}$ (countable union is measurable)
			\end{itemize}
			\item $\mu: \mathcal{F} \to [0, \infty]$ is a measure, i.e., a function satisfying:
			\begin{itemize}
				\item $\mu(\emptyset) = 0$
				\item Countable additivity: for any countable collection $\{A_1, A_2, \dots\}$ of disjoint sets in $\mathcal{F}$,
				\[
				\mu\Big(\bigcup_{i=1}^{\infty} A_i\Big) = \sum_{i=1}^{\infty} \mu(A_i)
				\]
			\end{itemize}
		\end{itemize}
		
	\end{frame}
	
	
	\begin{frame}{Definition (Measurable Space)}
		
		A \textbf{measurable space} is an ordered pair $(\Omega, \mathcal{F})$ where:
		
		\vspace{0.3cm}
		
		\begin{itemize}
			\item $\Omega$ is a set, called the sample space.
			\item $\mathcal{F}$ is a $\sigma$-algebra of subsets of $\Omega$, i.e., a collection of subsets that satisfies:
			\begin{itemize}
				\item $\Omega \in \mathcal{F}$
				\item if $A \in \mathcal{F}$, then $A^c \in \mathcal{F}$ (complement is measurable)
				\item if $A_1, A_2, \dots \in \mathcal{F}$, then $\bigcup_{i=1}^{\infty} A_i \in \mathcal{F}$ (countable union is measurable)
			\end{itemize}
			\item The elements of $\mathcal{F}$ are called \textbf{measurable sets}.
		\end{itemize}
		
	\end{frame}
	
	
	
	\begin{frame}{Probability Space}
		
		A probability space is a triple $(\Omega, \mathcal{F}, P)$ where:
		
		\begin{itemize}
			\item $\Omega$ is the sample space (all possible outcomes)
			\item $\mathcal{F}$ is a $\sigma$-algebra of subsets of $\Omega$ (the events)
			\item $P$ is a probability measure
		\end{itemize}
		
	\end{frame}
	
	
	
	\begin{frame}{Examples of Measures}
		
		\textbf{Length measure on $\mathbb{R}$:}  
		\[
		\mu((a,b)) = b - a
		\]
		
		\textbf{Counting measure on any set $\Omega$:}  
		\[
		\mu(A) = \text{number of elements in } A
		\]
		
		\textbf{Probability measure on a probability space $(\Omega, \mathcal{F}, P)$:}  
		\begin{itemize}
			\item $P(\Omega) = 1$
			\item $P$ satisfies countable additivity
		\end{itemize}
		
	\end{frame}
	
	
	\begin{frame}{Classical examples of Measures}
		\begin{block}{Lengths in $\mathbb{R}$}
			Length in $\mathbb{R}$: For any interval $I \in \mathbb{R}$ bounded by the endpoints $a,b$ , its length is given as \\ $\ell(I)$ = $|b-a|$.
			\vspace{1cm}
			\begin{exampleblock}{Additivity property of length}
				If an interval $I$ is a disjoint union of a finite family $\{I_k\}_{k=1}^{n}$ of intervals, then  $\ell(I) = \sum_{k=1}^{n} \ell(I_k)$.
			\end{exampleblock}
			
			
		\end{block}
	\end{frame}
	
	\begin{frame}{Classical examples of Measure}
		\begin{block}{Areas in $\mathbb{R}^2$}
			Given that $I, J$ are the intervals (or lengths) of any rectangle $A$, then \\ area$(A) = \ell(I)\ell(J)$.
			\vspace{1cm}
			\begin{exampleblock}{Additive property of Area}
				If $A$ is a rectangle of disjoint union of a finite family of rectangles $A_1, A_2, \dots, A_n$, then area$(A) = \sum_{k=1}^{n}$ area$(A_k)$.
			\end{exampleblock}
		\end{block}
	\end{frame}
	
	\begin{frame}{}
		\begin{block}{Volumes in $\mathbb{R}^3$}
			Any box in $\mathbb{R}^3$ of the form $A=I$x$J$x$K$, where $I, J, K$ are intervals in $\mathbb{R}$, will yield the set vol$(A) = \ell(I) \ell(J) \ell(K)$.
			\begin{itemize}
				\item The additive property of the volume is proved similarly.
			\end{itemize}
			
		\end{block}
		\vspace{1cm}
		\begin{block}{Probability}
			If the event $A \subset \Omega$, and $A$ is a disjoint union of a finite sequence of events $A_1, \dots, A_n$, then $\mathbb{P}(A) = \sum_{k=1}^{n} \mathbb{P}(A_k)$.
			
		\end{block}
	\end{frame}
	
	\begin{frame}{}
		
		\begin{exampleblock}{Note these similarities}
			All the above had the following.
			\begin{itemize}
				\item A non-empty set $M$ (i.e. $\mathbb{R}, \mathbb{R}^2, \mathbb{R}^3, \Omega)$.
				\item A family of subsets $S$ (i.e. intervals, rectangles, boxes, events).
				\item A functional $\mu: S \to \mathbb{R}_+ :=[0,+\infty)$ (length, area, etc.) with the following property:
				\begin{itemize}
					\item if $A \in S$ is a dsijoint union of a finite family $\{A_k\}_{k=1}^{n}$ of sets from $S$, then $\mu(A) = \sum_{k=1}^{n} \mu(A_k)$.
				\end{itemize}
			\end{itemize}
		\end{exampleblock}
	\end{frame}
	
	
	\begin{frame}{$\sigma$-additive measures}
		Let $M$ be a non-empty set and $S$ be a family of subsets of $M$. 
		\begin{block}{Definition}
			A functional $\mu: S \to \mathbb{R}_+$ is called a \textbf{$\sigma$-additive measure} if whenever
			\begin{itemize}
				\item a set $A \in S$ is a disjoint union of an at most countable sequence $\{A_k\}_{k=1}^{N}$ (where $N$ is either finite or $N = \infty$), then
				\item $\mu(A) = \sum_{k=1}^{N} \mu(A_k)$.
			\end{itemize}
		\end{block}
		
		\begin{block}{Remark}
			\begin{itemize}
				\item $\mu$ is a \textit{finitely additive measure} if this property holds for finite values of $N$.
				\item Every $\sigma$-additive measure is finitely additive, but the converse is not true.
			\end{itemize}
		\end{block}
	\end{frame}
	
	\begin{frame}{$\sigma$-subadditive measures}
		\begin{block}{Definition}
			A functional $\mu: \mathcal{S} \to \mathbb{R}_+$ is called $\sigma$-subadditive if whenever $A \subset \displaystyle \bigcup_{k=1}^{N} A_k$ where $A$ and $A_k$ are all elements of $\mathcal{S}$ and $N$ is either finite or infinite, \\
			$\mu(A) \leq \displaystyle \sum_{k=1}^{N} \mu(A_k)$.
			
			\begin{alertblock}{Note}
				If this property holds for finite values of $N$, then $\mu$ is called finitely subadditive.
			\end{alertblock}
		\end{block}
	\end{frame}
	
	\begin{frame}{Lemma 1.1}
		\begin{block}{}
			\begin{center}
				The \textit{length} is $\sigma$-subadditive.
			\end{center}
		\end{block}
	\end{frame}
	
	\begin{frame}{Proof}
		Let $I$, $\{I_k\}_{k=1}^{\infty}$ be intervals such that $I \subset \cup_{k=1}^{\infty} I_k$, we want to prove that \\
		$\ell(I) \leq \displaystyle \sum_{k=1}^{\infty} \ell(I_k)$.\\ \vspace{0.15cm}
		Let us fix some $\varepsilon > 0$ and choose a bounded closed interval $I' \subset I$ such that
		$\ell(I) \leq \ell(I')$ + $\varepsilon$.\\ \vspace{0.15cm}
		For any $k$, choose an open interval $I_{k}^{'} \supset I_k$ such that \\ $\ell(I'_k) \leq \ell(I_k)$ + $\frac{\varepsilon}{2^k}$.\\ \vspace{0.15cm}
		Then the bounded closed interval $I'$ is covered by a sequence $\{I'_k\}$ of open intervals. By the Borel-Lebesgue lemma, there is a finite subfamily $\{I'_{k_j}\}_{j=1}^{n}$ that also covers $I'$. 
		
	\end{frame}
	
\begin{frame}{Proof}
	It follows from the finite additivity of length that it is finitely subadditive. That is, 
	\[
	\ell(I') \leq \sum_{j} \ell(I'_{k_j}) \implies 
	\ell(I') \leq \sum_{k=1}^{\infty} \ell(I'_k).
	\]
	
	This yields 
	\[
	\ell(I) \leq \varepsilon + \sum_{k=1}^{\infty}\left(\ell(I_k) + \frac{\varepsilon}{2^k}\right) 
	= 2\varepsilon + \sum_{k=1}^{\infty} \ell(I_k).
	\]
	
	Since $\varepsilon > 0$ is arbitrary, letting $\varepsilon \to 0$ finishes the proof. $\square$
\end{frame}

	
	\begin{frame}{Theorem}
		\begin{block}{}
			The length is a $\sigma$-additive measure on the family of all bounded intervals in $\mathbb{R}$.
		\end{block}
	\end{frame}
	
	\begin{frame}{Proof}
		We need to prove that if
		$I = \displaystyle \bigsqcup_{k=1}^{\infty} I_k$, then $\ell(I) = \sum_{k=1}^{\infty} \ell(I_k)$.\\
		By the $\sigma$-subadditive lemma, we have $\ell(I) \leq \displaystyle \sum_{k=1}^{\infty} \ell(I_k)$, so we need to prove the opposite inequality. \\
		For a fixed $n \in \mathbb{N}$, we have \\ 
		$I \supset \displaystyle \bigsqcup_{k=1}^{n} I_k$.\\
		It follows from the finite additivity of length that \\
		$\ell(I) \ge \displaystyle \sum_{k=1}^{n} \ell(I_k)$.
	\end{frame}
	
	\begin{frame}{Proof}
		Letting $n \to \infty$, we obtain \\
		$\ell(I) \ge \displaystyle \sum_{k=1}^{\infty} \ell(I_k)$\\
		which finishes the proof.
	\end{frame}
	
	
	
		\begin{frame}{Outer Measure}
		Let $\Omega$ be a set and $R$ be an algebra of sets of $\Omega$. Suppose $\mu$ is a $\sigma$-additive measure on $R$. From the algebra property: if $A \in R$, then $A^c := \Omega \setminus A \in R$.
		\begin{block}{Definition}
			For any set $A \subset \Omega$, define its outer measure $\mu^*(A)$ by: \\
			$\mu^*(A) =$ inf $\Bigg \{\displaystyle \sum_{k=1}^{\infty} \mu(A_k) : A_k \in R$ and $A \subset \displaystyle \bigcup_{k=1}^{\infty} A_k \Bigg \}$ $\dots \dots \dots \dots (1)$
			\label{eq:energy}
		\end{block}
		
		\begin{block}{}
			In other words, we consider all coverings $\{A_k\}_{k=1}^{\infty}$ of $A$ by a sequence from the algebra $R$ and define $\mu^*(A)$ as the infimum of the sum of all $\mu(A_k)$, taken over all such coverings. 
		\end{block}
	\end{frame}
	
	
	\begin{frame}{Outer Measure}
		\begin{block}{Properties of outer measure}
			\begin{itemize}
				\item Null (empty) set: $\mu^*(\emptyset) = 0$
				\item Monotonicity: if $A \subset B \subset \Omega$, then $\mu^*(A) \leq \mu^*(B)$.
				\item Countable subadditivity ($\sigma$-subadditivity): For any countable collection of sets $\{A_n\}_{n=1}^{\infty} \subset \Omega$:\\
				$\mu^* \Bigg (\displaystyle \bigcup_{n=1}^{\infty} A_n \Bigg) \leq \displaystyle \sum_{n=1}^{\infty} \mu^*(A_n)$.
			\end{itemize}
		\end{block}
	\end{frame}
	
	\begin{frame}{Outer measure}
		\begin{block}{Lemma 1.2}
			For any set $A \subset \Omega$, $\mu^*(A) < \infty$ and if $A \in R$, then $\mu^*(A) = \mu(A)$.
		\end{block}
	\end{frame}
	
	\begin{frame}{Outer measure}
		\begin{block}{Proof}
			Note that $\emptyset \in R$ and $\mu(\emptyset) = 0$ because $\mu(\emptyset) = \mu(\emptyset \sqcup \emptyset) = \mu(\emptyset) + \mu(\emptyset)$. \\
			
			\vspace{0.3cm}
			For any set $A \subset \Omega$, consider a covering $\{A_k\} = \{\Omega, \emptyset, \emptyset, \dots\}$ of $A$. \\
			\vspace{0.3cm}
			Since $\Omega, \emptyset \in R$, it follows from (1) that \\ \vspace{0.2cm}
			$\mu^*(A) \le \mu(\Omega) + \mu(\emptyset) + \mu(\emptyset) + \dots = \mu(\Omega) < \infty$. \\
			
			\vspace{0.3cm}
			Assume now $A \in R$. Considering a covering $\{A_k\} = \{A, \emptyset, \emptyset, \dots\}$ \\ 
			\vspace{0.3cm}
			and using that $A, \emptyset \in R$, we obtain in the same way that \\ 
			\vspace{0.3cm}
			$\mu^*(A) \le \mu(A) + \mu(\emptyset) + \mu(\emptyset) + \dots = \mu(A)$. \\
			
		\end{block}
	\end{frame}
	
	\begin{frame}{Outer measure}
		\begin{block}{Continuation of Proof}
			On the other hand, for any sequence $\{A_k\}$ as in (1), we have by the \\
			\vspace{0.3cm}
			$\sigma$-subadditivity of $\mu$ that \\
			\vspace{0.3cm}
			$\sum_{k=1}^{\infty} \mu(A) \le \sum_{k=1}^{\infty} \mu(A_k)$. \\
			
			\vspace{0.3cm}
			Taking the infimum over all such sequences $\{A_k\}$, we obtain \\
			\vspace{0.3cm}
			$\mu(A) \le \mu^*(A)$, which together with the previous inequality yields \\
			\vspace{0.3cm}
			$\mu^*(A) = \mu(A)$. $\blacksquare$
		\end{block}    
	\end{frame}
	
	\begin{frame}{Outer measure}
		\begin{block}{Lemma 1.3}
			The outer measure $\mu^*$ is $\sigma$-subadditive on $2^\Omega$.
		\end{block}
	\end{frame}
	
	\begin{frame}{Outer measure}
		\begin{block}{Proof}
			We need to prove that if  
			$A \subset \bigcup_{k=1}^{\infty} A_k$ where
			$A$ and $A_k$ are subsets of $\Omega$, \\ \vspace{0.3cm}
			then $\mu^*(A) \le \sum_{k=1}^{\infty} \mu^*(A_k)$.\\ 
			\vspace{0.3cm}
			By the definition of $\mu^*$, for any set $A_k$ and any $\varepsilon > 0$ \\ \vspace{0.3cm}
			there exists a sequence $\{A_{k n}\}_{n=1}^{\infty}$ of sets from $R$ such that \\ \vspace{0.3cm}
			$A_k \subset \bigcup_{n=1}^{\infty} A_{k n}$ and $\mu^*(A_k) \ge \sum_{n=1}^{\infty} \mu(A_{k n}) - \frac{\varepsilon}{2^k}$. \\ \vspace{0.3cm}
			Adding these inequalities over all $k$, we obtain \\ \vspace{0.3cm} 
			$\sum_{k=1}^{\infty} \mu^*(A_k) \ge \sum_{k,n=1}^{\infty} \mu(A_{k n}) - \varepsilon$.
			
			
		\end{block}
	\end{frame}
	
	\begin{frame}{Outer measure}
		\begin{block}{Continuation of proof}
			On the other hand, by the inclusions $A \subset \bigcup_{k=1}^{\infty} A_k$ and $A_k \subset \bigcup_{n=1}^{\infty} A_{k n}$, \\
			\vspace{0.3cm}
			we get $A \subset \bigcup_{k,n=1}^{\infty} A_{k n}$. Since $A_{k n} \in R$, it follows from (1) that \\ \vspace{0.3cm}
			$\mu^*(A) \le \sum_{k,n=1}^{\infty} \mu(A_{k n})$. \\
			\vspace{0.3cm}
			Comparing with the previous inequality gives \\
			\vspace{0.3cm}
			$\mu^*(A) \le \sum_{k=1}^{\infty} \mu^*(A_k) + \varepsilon$. \\ \vspace{0.3cm}
			Since this holds for any $\varepsilon > 0$, it also holds for $\varepsilon = 0$, which completes the proof. $\blacksquare$
		\end{block}
	\end{frame}
	
	\begin{frame}{Symmetric Difference}
		\begin{block}{Definition}
			The symmetric difference of two sets $A, B \subset \Omega$ is the set $A \triangle B := (A \setminus B) \cup (B \setminus A) = (A \cup B) \setminus (A \cap B)$
			
			\begin{itemize}
				\item Clearly, $A \triangle B = B \triangle A$
				\item Also, $x \in A \triangle B$ if and only if $x$ belongs to exactly one of the sets $A, B$. That is, either $x \in A$ and $x \notin B$ or $ x \notin A$ and $ x \in B$.
			\end{itemize}
		\end{block}
	\end{frame}
	
	\begin{frame}{Symmetric Difference}
		\begin{block}{Lemma $1.4 (a)$}
			For arbitrary sets $A_1,A_2, B_1, B_2 \subset \Omega$, \\ \vspace{0.15cm}
			$(A_1 \circ A_2) \triangle (B_1 \circ B_2) \subset (A_1 \triangle B_1) \cup (A_2 \triangle B_2)$, \\ \vspace{0.15cm}
			where $\circ$ denotes any of the operations $\cup, \cap, \setminus.$
		\end{block}
		\vspace{1cm}
		\begin{block}{Lemma $1.4 (b)$}
			If $\mu^*$ is an outer measure on $\Omega$, then 
			$| \mu^*(A) - \mu^*(B)| \leq \mu^*(A \triangle B)$, \\
			for arbitrary sets $A, B \subset \Omega$.
		\end{block}
	\end{frame}
	
	
	\begin{frame}{Measurable Sets}
		We still consider $R$ to be an algebra on $\Omega$ and $\mu$ is a $\sigma$-additive measure on $R$, and holding the definition of $\mu^*$ by $(1)$.
		\begin{block}{Definition of Measurable sets}
			A set $A \subset \Omega$ is called measurable (with respect to the algebra $R$ and the measure $\mu$) if, for any $\varepsilon > 0$, there exist $B \in R$ such that \\ $\mu^*(A \triangle B) < \varepsilon$. 
		\end{block}
		
		\begin{block}{Definition}
			Let $\mu^*$ be an outer measure on a set $\Omega$.  
			A subset $E \subseteq \Omega$ is called \textbf{measurable} (with respect to $\mu^*$) if for every subset $A \subseteq \Omega$:
			\[
			\mu^*(A) = \mu^*(A \cap E) + \mu^*(A \cap E^c)
			\]
			where $E^c = \Omega \setminus E$ is the complement of $E$.
		\end{block}
	\end{frame}
	
	
	
	\begin{frame}{Measurable sets}
		\begin{block}{Examples of measurable sets}
			\begin{enumerate}
				\item Intervals in the real line. i.e. $(a, b), [a, b), [a,b], (-\infty, a), \text{or} (b, +\infty)$. \vspace{0.5cm}
				\item Finite sets and countable sets. i.e. $\{a, b, c\}, \mathbb{N}, \mathbb{Q}, $ etc. \vspace{0.5cm}
				\item Complements of measurable sets. i.e. $\mathbb{R} \setminus \mathbb{Q}$, etc
			\end{enumerate}
		\end{block}
	\end{frame}
	
	
	\begin{frame}{Measurable sets}
		\begin{block}{}
			Are all sets Measurable? Why?
		\end{block}	
	\end{frame}
	
	
	
	\begin{frame}{Non-Measurable sets}
		\begin{figure}[h!]
			\centering
			\includegraphics[width=0.9\textwidth]{non-measurable.png} % or .png
			\caption{examples of non-measurable sets.}
			\label{fig:non-measurable sets}
		\end{figure}
	\end{frame}
	
	
	
	\begin{frame}{Illustrative Examples of Non-Measurable Sets }
		
		\textbf{Vitali set ($\mathbb{R}$):}  
		\begin{itemize}
			\item Consider all real numbers between 0 and 1.
			\item Partition them into equivalence classes where numbers differ by a rational number.
			\item Pick one number from each class.
			\item The resulting set is the Vitali set.
			\item You cannot assign a consistent length to it, even though each piece “looks like a number.”
		\end{itemize}
		
		\vspace{0.3cm}
		
		\textbf{Banach–Tarski paradox (3D):}  
		\begin{itemize}
			\item Start with a solid 3D ball (like a basketball).
			\item Using very strange, non-physical pieces, it is possible to cut it into finitely many pieces and reassemble them into two balls of the same size.
			\item Each piece is non-measurable, meaning no volume can be consistently assigned.
		\end{itemize}
		
	\end{frame}
	
	
	\begin{frame}{Assignment}
		Illustrate the non-measurable Bernstein set of real numbers.
	\end{frame}
	\begin{frame}{Theorem}
		
		\begin{block}{Carathéodory's extension theorem}
			Let $R$ be an algebra on a set $\Omega$ and $\mu$ be a $\sigma$-additive measure on $R$. Denote by $\mathcal{M}$ the family of all measurable subsets of $\Omega$. Then the following is true:
			
			\begin{enumerate}
				\item $\mathcal{M}$ is a $\sigma$-algebra containing $R$.
				\item The restriction of $\mu^*$ on $\mathcal{M}$ is a $\sigma$-additive measure (that extends measure $\mu$ from $R$ to $\mathcal{M}$).
				\item If $\tilde{\mu}$ is a $\sigma$-additive measure defined on a $\sigma$-algebra $\mathcal{F}$ such that $R \subset \mathcal{F} \subset \mathcal{M}$, then $\tilde{\mu} = \mu^*$ on $\mathcal{M}$.
			\end{enumerate}
		\end{block}
	\end{frame}

	
	
	
	\begin{frame}{Caratheodory Extension Theorem Proofs}
		Claim 1: \textit{The family $\Omega$ of all measurable sets is an algebra containing $R$.}
		
		\begin{itemize}
			\item If $A \in R$, then $A$ is measurable because:
			\[
			\mu^*(A \triangle A) = \mu^*(\emptyset) = \mu(\emptyset) = 0,
			\] where $\mu^*(\emptyset) = \mu(\emptyset)$ by Lemma 1.2.
			\item Hence, $R \subset \Omega$ and the entire space $\Omega$ is a measurable set. 
		\end{itemize}
		
		\textbf{To verify $\mathcal{M}$ is an algebra:} Show that for $A_1,A_2 \in \mathcal{M}$, both $A_1 \cup A_2$ and $A_1 \setminus A_2$ are measurable.
		
		\begin{itemize}
			\item Let $A = A_1 \cup A_2$. For any $\varepsilon>0$, there exist $B_1,B_2 \in R$ such that
			\[
			\mu^*(A_1 \triangle B_1) < \varepsilon, \quad \mu^*(A_2 \triangle B_2) < \varepsilon
			\]
			\item Set $B = B_1 \cup B_2 \in R$. 
		\end{itemize}
		
		\end{frame}
		
		
\begin{frame}{Proof}
	\begin{itemize}
		\item Then by Lemmas 1.4 and 1.3 respectively:
		\begin{equation*}
			A \triangle B \subset (A_1 \triangle B_1) \cup (A_2 \triangle B_2)
		\end{equation*} and 
		\begin{equation*}
			\mu^*(A \triangle B) \le \mu^*(A_1 \triangle B_1) + \mu^*(A_2 \triangle B_2) < 2\varepsilon
		\end{equation*}
		\item Since $\varepsilon>0$ is arbitrary and $B \in R$ $A$ is measurable. Similarly, $A_1 \setminus A_2 \in \mathcal{M}$.
	\end{itemize}
\end{frame}

	
	\begin{frame}{Caratheodory Extension Theorem Proofs}
		Claim 2: \textit{\(\sigma\)-additivity of \(\mu^*\) on \(\mathcal{M}\)
		\(\mu^*\) is \(\sigma\)-additive on \(\mathcal{M}\).}  \\
	\vspace{0.3cm}
	
		Since \(\mathcal{M}\) is an algebra and \(\mu^*\) is \(\sigma\)-subadditive by Lemma 1.5, it suffices to prove that \(\mu^*\) is finitely additive on \(\mathcal{M}\) (see Exercise 9).  
		
		Let us prove that, for any two disjoint measurable sets \(A_1\) and \(A_2\), we have
		\[
		\mu^*(A) = \mu^*(A_1) + \mu^*(A_2)
		\]
		where \(A = A_1 \cup A_2\). By Lemma 1.5, we have the inequality
		\[
		\mu^*(A) \le \mu^*(A_1) + \mu^*(A_2)
		\]
		so that we are left to prove the opposite inequality
		\[
		\mu^*(A) \ge \mu^*(A_1) + \mu^*(A_2).
		\]
		

	\end{frame}
	
	
	\begin{frame}{Proof}
		For any \(\varepsilon>0\), there are sets \(B_1, B_2 \in \mathbb{R}\) such that (1.19) holds.  
		Set \(B = B_1 \cup B_2 \in \mathbb{R}\) and apply Lemma 1.6, which says that
		\[
		|\mu^*(A) - \mu^*(B)| \le \mu^*(A \triangle B) < 2 \varepsilon,
		\]
		where in the last inequality we have used (1.20). In particular,
		\[
		\mu^*(A) \ge \mu^*(B) - 2 \varepsilon.
		\]
		On the other hand, since \(B \in \mathbb{R}\), we have by Lemma 1.4 and the additivity of \(\mu\) on \(\mathbb{R}\) that
		\[
		\mu^*(B) = \mu(B) = \mu(B_1 \cup B_2) = \mu(B_1) + \mu(B_2) - \mu(B_1 \cap B_2).
		\]
		\end{frame}
		
		\begin{frame}{Proof}
			Next, we estimate \(\mu(B_i)\) from below via \(\mu^*(A_i)\) and show that \(\mu(B_1 \cap B_2)\) is small enough.  
			
			Indeed, using (1.19) and Lemma 1.6, we obtain, for any \(i = 1,2\),
			\[
			|\mu^*(A_i) - \mu^*(B_i)| \le \mu^*(A_i \triangle B_i) < \varepsilon,
			\]
			whence
			\[
			\mu(B_1) \ge \mu^*(A_1) - \varepsilon \quad \text{and} \quad \mu(B_2) \ge \mu^*(A_2) - \varepsilon.
			\]
			
			On the other hand, by Lemma 1.6 and using \(A_1 \cap A_2 = \emptyset\), we obtain
			\[
			B_1 \cap B_2 = (A_1 \cap A_2) \triangle (B_1 \cap B_2) \subset (A_1 \triangle B_1) \cup (A_2 \triangle B_2),
			\]
			whence by (1.20)
			\[
			\mu(B_1 \cap B_2) = \mu^*(B_1 \cap B_2) \le \mu^*(A_1 \triangle B_1) + \mu^*(A_2 \triangle B_2) < 2\varepsilon.
			\]
			
			It follows from (1.21)–(1.24) that
			\[
			\mu^*(A) \ge (\mu^*(A_1) - \varepsilon) + (\mu^*(A_2) - \varepsilon) - 2\varepsilon - 2\varepsilon = \mu^*(A_1) + \mu^*(A_2) - 6\varepsilon.
			\]
			
			Letting \(\varepsilon \to 0\), we finish the proof.
		\end{frame}
		
		
		\begin{frame}{Discuss}
			\Large Prove the following claims: \\
			\begin{enumerate}
				\item $\mathcal{M}$ is $\sigma$-algebra
				\vspace{1cm}
				\item If \(\Sigma\) is a \(\sigma\)-algebra such that
				\[
				\mathbb{R} \subset \Sigma \subset \Omega,
				\]
				and \(\tilde{\mu}\) is a \(\sigma\)-additive measure on \(\Sigma\) such that \(\tilde{\mu} = \mu\) on \(\mathbb{R}\). Then \(\tilde{\mu} = \mu^*\) on \(\Sigma\).  
				
				Prove that
				\[
				\tilde{\mu}(A) = \mu^*(A) \quad \text{for any } A \in \Sigma.
				\]
				
			\end{enumerate}
		\end{frame}
		
		\begin{frame}{$\sigma$-finite measures}
		\begin{block}{Definition}
			Let $\Omega$ be a non-empty set and $S$ be a family of subsets of $\Omega$.\\
			A functional $\mu : S \to [0, +\infty]$ is called a \textit{measure} if, for all sets $A, A_k \in S$ such that $A = \bigsqcup_{k=1}^{N} A_k$ (where N is either finite or infinite), we have \\
			$\mu(A) = \displaystyle \sum_{k=1}^{N} \mu(A_k)$.
		\end{block}
		
		\begin{block}{}
			Hence, a measure is always $\sigma$-additive.
		\end{block}
	\end{frame}
	
	
	\begin{frame}{Examples of \(\sigma\)-finite Measures}
		
		\textbf{Lebesgue measure on the real line.}  
		Let \(\mu\) be the usual length on \(\mathbb{R}\).  
		The measure of \(\mathbb{R}\) is infinite, but
		\[
		\mathbb{R} = \bigcup_{n=1}^{\infty} [-n, n].
		\]
		Each interval \([-n, n]\) has finite length \(2n\).  
		Therefore, Lebesgue measure on \(\mathbb{R}\) is \(\sigma\)-finite.
		
		\vspace{0.3cm}
		
		\textbf{Counting measure on the integers.}  
		Let \(X = \mathbb{Z}\) and define \(\mu(A)\) to be the number of elements in \(A\).  
		The set \(\mathbb{Z}\) is infinite, but
		\[
		\mathbb{Z} = \bigcup_{n \in \mathbb{Z}} \{n\}.
		\]
		Each singleton \(\{n\}\) has measure \(1\), which is finite.  
		Hence, the counting measure on \(\mathbb{Z}\) is \(\sigma\)-finite.
		
		\vspace{0.3cm}
		
		\end{frame}
		
		\begin{frame}
			\textbf{Area measure on the plane.}  
			Let \(\mu\) be the usual area measure on \(\mathbb{R}^2\).  
			The plane has infinite area, but
			\[
			\mathbb{R}^2 = \bigcup_{n=1}^{\infty} [-n, n] \times [-n, n].
			\]
			Each square has finite area \((2n)^2\).  
			Thus, the area measure on \(\mathbb{R}^2\) is \(\sigma\)-finite.
		\end{frame}	
		
	
	

	


	
	
	
	
	
	
	
	
	
	
	
	
	
	
	
	
	
	
	
	
	
	
	
	\end{document}