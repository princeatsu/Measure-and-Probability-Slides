\documentclass{beamer}

% ========== THEME & PACKAGES ==========
\usetheme{Madrid}
\usepackage{booktabs}
\usepackage{array}
\usepackage{graphicx}
\usepackage{xcolor}

% ========== PRESENTATION INFO ==========
\title[Measure Theory]{Probability and Measure Theory}
\author{Gabriel Asare Okyere (PhD)}
\institute[]{Department of Statistics and Actuarial Science, KNUST, Kumasi, Ghana}
\date{\today}

\setbeamertemplate{navigation symbols}{} % remove navigation icons

% ========== FOOTLINE COLORS ==========
\setbeamercolor{footline-left}{bg=blue!70!black, fg=white}
\setbeamercolor{footline-middle}{bg=blue!60!black, fg=white}
\setbeamercolor{footline-right}{bg=blue!50!black, fg=white}

% ========== CUSTOM FOOTLINE ==========
\setbeamertemplate{footline}{%
	\leavevmode%
	\hbox{%
		% --- Left: Author / Institution ---
		\begin{beamercolorbox}[wd=.33\paperwidth,ht=2.5ex,dp=1ex,center]{footline-left}%
			\textbf{Prof. Okyere (KNUST)}
		\end{beamercolorbox}%
		% --- Middle: Course code ---
		\begin{beamercolorbox}[wd=.33\paperwidth,ht=2.5ex,dp=1ex,center]{footline-middle}%
			\textbf{MSTAT 551}
		\end{beamercolorbox}%
		% --- Right: Year / Slide number / Logo ---
		\begin{beamercolorbox}[wd=.34\paperwidth,ht=2.5ex,dp=1ex,right]{footline-right}%
			\textbf{2026 \hspace{1em} \insertframenumber{} / \inserttotalframenumber}%
			\hspace{0.5em}%
			\raisebox{-0.5ex}{\includegraphics[height=0.6cm]{knust_logo.jpeg}}%
			\hspace{0.5em}%
		\end{beamercolorbox}%
	}%
	\vskip0pt%
}

% ===== TITLE INFORMATION =====
\title{MSTAT 551: Probability and Measure Theory}
\author{\Huge{Gabriel Asare Okyere (PhD)}}

\institute{\Large{Department of Statistics and Actuarial Science, KNUST.}}
\date{January 2026}

% ===== DOCUMENT START =====
\begin{document}
	\begin{frame}
		\titlepage
		
	\end{frame}
	
	\begin{frame}{Learning objectives}
		\begin{block}{}
			At the end of this section, learners will be able to:
			\begin{itemize}
				\item Explain the need for structured collections of sets in measure and probability theory.
				\item Construct rings and $\sigma$-algebras generated by given families of sets.
				\item Illustrate semi-rings, rings, algebras, and $\sigma$-algebras with concrete examples.
				\item Distinguish between semi-rings, rings, algebras, and $\sigma$-algebras using their closure properties.
				\item Analyze how increasing closure requirements lead from semi-rings to $\sigma$-algebras.
			\end{itemize}
			
		\end{block}
	\end{frame}




\begin{frame}{Towards Algebra -- The Idea}
	So far, we have worked with individual sets such as intervals and simple events. \\
	To define length, probability, or measure, we need to work with collections of sets, not just single sets.\\
	We want a collection of sets with the following properties:
	\begin{itemize}
		\item the empty set should be included
		\item complements should remain inside the collection
		\item unions of sets should stay inside the collection
	\end{itemize}
	
	However, requiring all these properties at once is often too strong at the beginning.\\
	Instead, we start with very simple building blocks, such as intervals, and gradually add structure.
\end{frame}

\begin{frame}{Semi-ring of Sets}
	\begin{block}{Definition (Semi-ring)}
		A family $\mathcal{S}$ of subsets of $\Omega$ is called a semi-ring if:
		\begin{itemize}
			\item $\emptyset \in \mathcal{S}$
			\item If $A, B \in \mathcal{S}$, then $A \cap B \in \mathcal{S}$
			\item If $A, B \in \mathcal{S}$, then $A \setminus B$ is a disjoint union of a finite family of sets from $\mathcal{S}$
		\end{itemize}
	\end{block}
\end{frame}


\begin{frame}{Semi-Ring}
	\begin{block}{Example}
		Consider the set of half-open intervals in $\mathbb{R}$:
		\begin{equation*}
			\mathcal{S} = \{ [a, b) \mid a < b, \ a, b \in \mathbb{R} \}.
		\end{equation*}
	\end{block}    
\end{frame}

\begin{frame}{Semi-ring}
	\begin{exampleblock}{Proof that $\mathcal{S}$ is a semi-ring.}
		\begin{itemize}
			\item Take $[a, a) = \emptyset$, so $\emptyset \in \mathcal{S}$.
			\item The intersection $[a, b) \cap [c, d) = [\max(a,c), \min(b,d)) \in \mathcal{S}$.
			\item Suppose that $[a, b) \subseteq [c, d)$. The difference $[a, b) \setminus [c, d)$ is a finite union of disjoint sets from $\mathcal{S}$. \\
			That is, $[a, b) \setminus [c, d) = [c, a) \cup [b, d)$.
		\end{itemize}
	\end{exampleblock}
	
	\begin{exampleblock}{Other examples of semi-ring}
		\begin{itemize}
			\item All bounded intervals in $\mathbb{R}$ are semi-rings.
			\item Finite unions of half-open intervals in $\mathbb{R}$.
			\item Finite subsets of any set.
		\end{itemize}
	\end{exampleblock}
	
	\begin{alertblock}{}
		\textbf{Prove the 3 examples above.}
	\end{alertblock}
	
\end{frame}

\begin{frame}{Ring}
	\begin{block}{Definition of a Ring}
		A family $\mathcal{R}$ of subsets of $\Omega$ is called a ring if
		\begin{itemize}
			\item $\mathcal{R}$ contains $\emptyset$
			\item $A, B \in \mathcal{R} \implies A \cup B \in \mathcal{R},$ and $A \setminus B \in \mathcal{R}$.
		\end{itemize} \vspace{0.25cm}
		It follows that the intersection $A \cap B$ belongs to $\mathcal{R}$ because \\ $A \cap B = B \setminus (B \setminus A)$ is also in $\mathcal{R}$. \\
		Also, it follows that a ring is a semi-ring.
	\end{block}
\end{frame}

\begin{frame}{Ring}
	\begin{block}{Example}
		Let $\Omega = \{1, 2, 3, 4,5, \dots\}$ and let $\mathcal{R} = \{ \text{All finite subsets of $\Omega$}\}$.
	\end{block}
	\begin{exampleblock}{Proof that $\mathcal{R}$ is a ring}
		\begin{itemize}
			\item $\emptyset \subseteq \Omega$ and is finite, therefore $\emptyset \in \mathcal{R}$
			\item $\{1, 3\} \cup \{2, 4, 5\} = \{1, 2, 3, 4, 5\} \in \mathcal{R}$.
			\item $\{1, 2, 3\} \setminus \{2\} = \{1, 3\} \in \mathcal{R}$.
		\end{itemize}
	\end{exampleblock}
	\begin{alertblock}{Other examples of a ring}
		\begin{itemize}
			\item $\mathcal{R} = \{\text{all finite subsets of } \mathbb{N}\} \text{ is a ring.}$
			\item $\mathcal{R} = \{[a, b) \subseteq \mathbb{R} | a, b \in \mathbb{Q}, a \leq b\}$. That is, all intervals of this form on the real line with rational endpoints.
		\end{itemize}
		
	\end{alertblock}
\end{frame}


\begin{frame}{$\sigma$-ring}
	\begin{block}{Definition of $\sigma$-ring}
		A ring $\mathcal{R}$ is called a $\sigma$-ring if the union of any countable family $\{A_k\}_{k=1}^{\infty}$ of sets from $\mathcal{R}$ is also in $\mathcal{R}$. That is:
		\begin{itemize}
			\item Closure under unions: If $A_1, A_2, A_3, \dots \in \mathcal{R}$, then \\
			$\displaystyle \bigcup_{n=1}^{\infty} A_n \in \mathcal{R}$. 
			\item Closure under difference: If $A, B \in \mathcal{R}$, then $A \setminus B \in \mathcal{R}$.
		\end{itemize}
	\end{block}
	
	\begin{exampleblock}{}
		\begin{itemize}
			\item It follows that the intersection $A = \bigcap_{k} A_k$ is also in $\mathcal{R}$.
			\item Let $B$ be any of the sets $A_k$ so that $B \supset A$, then \\
			$A = B \setminus (B \setminus A) = B \setminus \left (\bigcup_{k} (B \setminus A_k) \right ) \in \mathcal{R}$.
		\end{itemize} 
	\end{exampleblock}
\end{frame}


\begin{frame}{$\sigma$-ring}
	\begin{block}{Examples of $\sigma$-ring}
		\begin{itemize}
			\item The power set of a set is a $\sigma$-ring. \\
		\end{itemize}
	\end{block}
	\begin{exampleblock}
			
			\textbf{Proof:}\\
			i. Empty set \\[4pt]
			
			By definition of the power set, every subset of $\Omega$ belongs to $\mathcal{P}(\Omega)$. \\
			Since $\emptyset \subseteq \Omega$, we have
			\[
			\emptyset \in \mathcal{P}(\Omega).
			\]
			
			ii. Closed under set difference \\[4pt]
			
			Let $A,B \in \mathcal{P}(\Omega)$.
			Then $A \subseteq \Omega$ and $B \subseteq \Omega$.
			
			The difference $A \setminus B$ is also a subset of $\Omega$.
			Therefore,
			\[
			A \setminus B \in \mathcal{P}(\Omega).
			\]
			
\end{exampleblock}
\end{frame}


\begin{frame}
	\begin{exampleblock}{}
		iii. Closed under countable unions \\[4pt]
		
		Let $A_1, A_2, A_3, \ldots$ be any sequence of sets in $\mathcal{P}(\Omega)$.
		Each $A_n$ is a subset of $\Omega$.
		
		The union $\bigcup_{n} A_n$ is also a subset of $\Omega$.
		Hence,
		\[
		\bigcup_{n} A_n \in \mathcal{P}(\Omega).
		\]
		
	\end{exampleblock}
	
	\vspace{0.5cm}
	
	\begin{block}{Other examples of a $\sigma$-ring}
		\begin{enumerate}
			\item All countable subsets of a set.
			\item All finite subsets of a set.
			\item The $\sigma$-ring generated by a semi-ring.
		\end{enumerate}
	\end{block}
\end{frame}


\begin{frame}{A ring generated by a family of sets}
	\begin{block}{Definition}
		Let $S \subseteq 2^\Omega$ be a family of subsets of $\Omega$. Define
		\[
		\mathcal{R}(S) = \bigcap \big\{ \mathcal{R} : \mathcal{R} \text{ is a ring of subsets of } \Omega \text{ and } S \subseteq \mathcal{R} \big\}.
		\]
		
		Then $\mathcal{R}(S)$ is called the \emph{ring generated by $S$}. It is the \emph{smallest ring} containing $S$.  
		
		\vspace{0.5cm}
		
		\begin{exampleblock}{Intuition}
			If $\Omega = \mathbb{R}$ and $S$ is the family of all intervals $[a,b]$, then $\mathcal{R}(S)$ is the ring of \emph{finite unions of intervals}.  
			
			The minimal ring contains exactly what you need to satisfy the ring properties, and nothing extra.
		\end{exampleblock}

	\end{block}
	
\end{frame}


\begin{frame}{Construction}
	\textbf{Setup: } Let $\Omega = \{1,2,3,4\}$ and let the family of sets be
	$S = \{\{1\}, \{2\}\}.$
	
	\textbf{Step 1: Build a ring} \\
	
	A ring must be closed under \emph{union} and \emph{difference}. Starting with $S$:  
	
	\emph{Unions:} 
	\[
	\{1\} \cup \{2\} = \{1,2\}
	\]
	
	\emph{Differences:} 
	\[
	\{1,2\} \setminus \{1\} = \{2\}, \quad
	\{1,2\} \setminus \{2\} = \{1\}
	\]
	
	\textbf{Step 2: Include empty set} \\
	
	Every ring contains the empty set:
	\[
	\emptyset \in \mathcal{R}(S)
	\]
	
	\textbf{Step 3: Minimal ring} \\
	
	After including all unions and differences, the smallest ring containing $S$ is:
	\[
	\mathcal{R}(S) = \{\emptyset, \{1\}, \{2\}, \{1,2\}\}.
	\]
	
	
\end{frame}


\begin{frame}{Group Assignment}
	 Analyze the difference and similarities between a semi-ring, ring, and a $\sigma$-ring. Add examples.
	 \vspace{0.15cm}
	 \centering
	 \includegraphics[width=0.9\textwidth]{group.jpg}
	 
\end{frame}


\begin{frame}{Definition of a Semi-Algebra of Sets}
	
	\begin{block}{Semi-Algebra of Sets}
		Let $\Omega$ be a nonempty set. A collection $\mathcal{S} \subseteq \mathcal{P}(\Omega)$ is called a \textbf{semi-algebra} if it satisfies:
		
		\begin{enumerate}
			\item $\varnothing \in \mathcal{S}$
			\item \textbf{Closure under finite intersections:} if $A, B \in \mathcal{S}$, then
			\[
			A \cap B \in \mathcal{S}
			\]
			\item \textbf{Complement is a finite disjoint union:} for every $A \in \mathcal{S}$, there exist pairwise disjoint sets $A_1, \dots, A_n \in \mathcal{S}$ such that
			\[
			A^c = \bigcup_{k=1}^n A_k
			\]
		\end{enumerate}
	\end{block}

\end{frame}


\begin{frame}{Example of a Semi-Algebra}
	
	\begin{block}{Semi-Algebra of Intervals on $\mathbb{R}$}
		Let 
		\[
		\mathcal{S} = \{ [a,b) \mid a \le b, \ a,b \in \mathbb{R} \}.
		\]
		\textbf{Claim:} $\mathcal{S}$ is a semi-algebra on $\mathbb{R}$.
		
		\textbf{Proof:}
		\begin{enumerate}
			\item \textbf{Empty set:} 
			\[
			[a,a) = \varnothing \in \mathcal{S}.
			\]
			
			\item \textbf{Closure under finite intersections:} 
			Let $[a,b), [c,d) \in \mathcal{S}$. Then
			\[
			[a,b) \cap [c,d) = [\max(a,c), \min(b,d)),
			\]
			which is either empty or an interval in $\mathcal{S}$.
			
			\item \textbf{Complement as a finite disjoint union:} 
			For any $[a,b) \in \mathcal{S}$,
			\[
			[a,b)^c = (-\infty, a) \cup [b, \infty),
			\]
			
		\end{enumerate}
	\end{block}
\end{frame}


\begin{frame}{Example of a Semi-algebra}
	\begin{block}{}
		which is a union of two disjoint sets that can be expressed as intervals of the same type.\\
			\medskip
			
			Therefore, $\mathcal{S}$ satisfies all conditions of a semi-algebra.
	\end{block}
		
	\medskip
	
	\textbf{Plain explanation:} a semi-algebra allows intersections, but not necessarily complements. However, the complement of any set can be written as a finite union of disjoint sets in the collection.
\end{frame}


\begin{frame}{Simple Examples of Semi-Algebras}
	
	\begin{enumerate}
		\item \textbf{Intervals starting from 0 on the real line:} 
		\[
		\mathcal{S} = \{ [0, a) \mid a \ge 0 \}.
		\]
		\begin{itemize}
			\item Finite intersections: $[0,a) \cap [0,b) = [0, \min(a,b)) \in \mathcal{S}$
			\item Complement: $[0,a)^c = [a, \infty)$, a single interval
		\end{itemize}
		
		\item \textbf{Single-element and empty sets in a finite set:} 
		Let $\Omega = \{1,2,3\}$ and
		\[
		\mathcal{S} = \{\varnothing, \{1\}, \{2\}, \{3\}\}.
		\]
		\begin{itemize}
			\item Finite intersections: intersection of any sets in $\mathcal{S}$ is still in $\mathcal{S}$
			\item Complement: e.g., $\{1\}^c = \{2,3\}$, which can be written as a union of disjoint singletons
		\end{itemize}
	\end{enumerate}
	
\end{frame}



\begin{frame}{Algebra}
	\begin{block}{Definition of an Algebra of Sets}
		Let $\Omega$ be a non-empty set.\\
		A collection of subsets of $\Omega$ is called an algebra $\mathcal{A}$, if it satisfies the following conditions:
		\begin{enumerate}
			\item $\Omega \in \mathcal{A}$
			\item $A \in \mathcal{A} \implies A^c \in \mathcal{A}$
			\item If $A_1, A_2, \dots, A_n \in \mathcal{A}$, then  $\bigcup_{i=1}^{n} A_i \in \mathcal{A}$
		\end{enumerate}
	\end{block}
	\vspace{1cm}
	
	\begin{exampleblock}{Note}
		A ring containing $\Omega$ is called an algebra.
	\end{exampleblock}
	
\end{frame}

\begin{frame}{Algebra}
	\begin{exampleblock}{Example}
		Given $\Omega = \{1, 2, 3\}$. Construct the algebra $\mathcal{A}$.
	\end{exampleblock}
	
\end{frame}

\begin{frame}{Algebra}
	\begin{exampleblock}{Solution}
		\begin{itemize}
			\item $\Omega = \{1, 2, 3\}$
			\item $\mathcal{A} = \{\emptyset, \Omega, \{2\}, \{1,3\}\}$
			\item $\mathcal{A} = \{\emptyset, \Omega, \{1\}, \{2,3\}\}$
			\item $\mathcal{A} = \{\emptyset, \Omega, \{3\}, \{1,2\}\}$
		\end{itemize}
		Clearly, any of the $\mathcal{A}$ satisfies all three properties of an algebra. 
	\end{exampleblock} \vspace{1cm}
	
	\begin{alertblock}{Note}
		The power set of $\Omega$, denoted by $\mathcal{P}(\Omega)$ or $2^\Omega$ is an algebra.
	\end{alertblock}
\end{frame}


\begin{frame}{$\sigma$-Algebra}
	\begin{block}{Definition of $\sigma$-Algebra}
		A collection of sets $\mathcal{F}$ is called a $\sigma$-algebra if it is an algebra and satisfies the property:
		\begin{itemize}
			\item If $A_n \in \mathcal{F}$ for $n \ge 1$, then $\bigcup_{n \ge 1} A_n \in \mathcal{F}$.
		\end{itemize}
		That is, a $\sigma$-algebra is an algebra and is closed under countable unions.
	\end{block}
	
	\begin{block}{}
		\begin{itemize}
			\item[(i)] $\Omega \in \mathcal{F}$,
			\item[(ii)] $A \in \mathcal{F} \implies A^c \in \mathcal{F}$,
			\item[(iii)] $A_1, A_2, \dots \in \mathcal{F} \implies \bigcup_{j=1}^\infty A_j \in \mathcal{F}$
		\end{itemize}
	\end{block}
\end{frame}


\begin{frame}{$\sigma$-Algebra}
	\begin{block}{Some general, simple examples of $\sigma$-algebras}
		\begin{enumerate}
			\item $\mathcal{F} = \{\emptyset, \Omega\}$ --- \textit{trivial $\sigma$-algebra}
			\item $\mathcal{F} = \{\text{all subsets of } \Omega\}$ --- The largest $\sigma$-algebra
			\item Let $\mathcal{A} = \{A\} \subset \Omega$, then \\ 
			$\sigma(\mathcal{A}) = \{\emptyset, A, A^c, \Omega\}$
		\end{enumerate}
	\end{block}
\end{frame}


\begin{frame}{A $\sigma$-algebra generated by a set}
	\begin{block}{Definition}
		Let $S \subseteq P(\Omega)$ be a family of subsets of $\Omega$. Define
		\[
		\sigma(S) = \bigcap \Big\{ \mathcal{F} \subseteq 2^\Omega : \mathcal{F} \text{ is a sigma-algebra and } S \subseteq A \Big\}.
		\]
		
		Then $\sigma(S)$ is called the \emph{sigma-algebra generated by $S$}. It is the \emph{smallest sigma-algebra} containing $S$.
		
	\end{block}
\end{frame}


\begin{frame}{Construction of a $\sigma$-algebra generated by a set}
	\textbf{Step 1: Start with the set} \\
	
	Let $\Omega = \{1,2,3\} \quad \text{and} \quad S = \{\{1\}\}$
	
	as your starting set.
	
	\textbf{Step 2: Include complements} \\  
	
	The complement of $\{1\}$ in $\Omega$ is $\{2,3\}.$
	
	Always include the empty set $\emptyset$ and the whole set $\Omega$.  
	
	So now we have: $\emptyset, \{1\}, \{2,3\}, \Omega$
	
	\textbf{Step 3: Include countable unions and intersections} \\
	
	Check unions and intersections of all included sets:
	\begin{center}
		$\{1\} \cup \{2,3\} = \Omega \quad \text{(already included)}$
	\end{center}
	
	\begin{center}
		$\{1\} \cap \{2,3\} = \emptyset \quad \text{(already included)}$
	\end{center}
	
	
	All other unions and intersections are also already included. \\
	\vspace{0.15cm}
	
	The sigma-algebra generated by $S$ is:
	\[
	\sigma(S) = \{\emptyset, \{1\}, \{2,3\}, \{1,2,3\}\}.
	\]
	
\end{frame}

\begin{frame}{$\sigma$-Algebra}
	\begin{block}{Discussion}
		Given $\Omega = \{HH, TH, HT, TT\}$. Construct the following on $\Omega$.
		\begin{itemize}
			\item $\sigma(\{HH, TT\})$
			\item $\sigma(\{TH\})$
			\item $\sigma(\{\emptyset\})$
			\item $\sigma(\{TH\}, \{HT\})$
		\end{itemize}
	\end{block}
	
	\vspace{1cm}
	
	\begin{exampleblock}{Note}
		A $\sigma$-ring on $\Omega$ containing $\Omega$ is called a $\sigma$-algebra.
	\end{exampleblock}
\end{frame}



\begin{frame}{Is every \textit{Algebra} a $\sigma$-Algebra? Is the converse true?}

	\centering
	\includegraphics[width=0.95\textwidth]{discuss.jpg}
\end{frame}


\begin{frame}{Hierarchy of Set Systems}
	
	\begin{block}{Important Takeaway}
		Every semi-algebra can generate an algebra, and every algebra is a $\sigma$-algebra if it is closed under countable unions.  
		
		\medskip
		Thus, a $\sigma$-algebra is stronger than an algebra, which is stronger than a semi-algebra, because each adds more closure properties:
		
		\[
		\begin{aligned}
			&\text{Semi-Algebra: finite intersections} \\
			&\rightarrow \text{Algebra: complements + finite unions} \\
			&\rightarrow \text{$\sigma$-Algebra: countable unions}
		\end{aligned}
		\]
	\end{block}
	
\end{frame}


\end{document}