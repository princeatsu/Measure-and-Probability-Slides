\documentclass{beamer}

% ========== THEME & PACKAGES ==========
\usetheme{Madrid}
\usepackage{booktabs}
\usepackage{array}
\usepackage{graphicx}
\usepackage{xcolor}

% ========== PRESENTATION INFO ==========
\title[Measure Theory]{Measure and Probability Theory}
\author{Gabriel Asare Okyere (PhD)}
\institute[]{Department of Statistics and Actuarial Science, KNUST, Kumasi, Ghana}
\date{\today}

\setbeamertemplate{navigation symbols}{} % remove navigation icons

% ========== FOOTLINE COLORS ==========
\setbeamercolor{footline-left}{bg=blue!70!black, fg=white}
\setbeamercolor{footline-middle}{bg=blue!60!black, fg=white}
\setbeamercolor{footline-right}{bg=blue!50!black, fg=white}

% ========== CUSTOM FOOTLINE ==========
\setbeamertemplate{footline}{%
	\leavevmode%
	\hbox{%
		% --- Left: Author / Institution ---
		\begin{beamercolorbox}[wd=.33\paperwidth,ht=2.5ex,dp=1ex,center]{footline-left}%
			\textbf{Prof. Okyere (KNUST)}
		\end{beamercolorbox}%
		% --- Middle: Course code ---
		\begin{beamercolorbox}[wd=.33\paperwidth,ht=2.5ex,dp=1ex,center]{footline-middle}%
			\textbf{STAT 751}
		\end{beamercolorbox}%
		% --- Right: Year / Slide number / Logo ---
		\begin{beamercolorbox}[wd=.34\paperwidth,ht=2.5ex,dp=1ex,right]{footline-right}%
			\textbf{2026 \hspace{1em} \insertframenumber{} / \inserttotalframenumber}%
			\hspace{0.5em}%
			\raisebox{-0.5ex}{\includegraphics[height=0.6cm]{knust_logo.jpeg}}%
			\hspace{0.5em}%
		\end{beamercolorbox}%
	}%
	\vskip0pt%
}

% ===== TITLE INFORMATION =====
\title{STAT 751: Measure and Probability Theory}
\author{\Huge{Gabriel Asare Okyere (PhD)}}

\institute{\Large{Department of Statistics and Actuarial Science, KNUST.}}
\date{January 2026}

% ===== DOCUMENT START =====
\begin{document}
	\begin{frame}
		\titlepage
		
	\end{frame}
	
	\begin{frame}{Learning objectives}
		\begin{block}{}
			At the end of this section, learners will be able to:
			\begin{itemize}
				\item Define and explain the concepts of measure, measurable spaces, measure spaces, and probability spaces, including their components and properties
				\item Illustrate classical examples of measures, such as length, area, volume, counting, and probability measures, and apply them to practical scenarios
				\item Differentiate between finite additivity, $\sigma$-additivity, and $\sigma$-subadditivity, and demonstrate their relationships with examples
				\item Apply the principles of countable additivity and subadditivity to sequences of sets and prove key properties of measures, including length on bounded intervals
				\item Analyze and evaluate the extension of simple measures to complex or infinite sets, emphasizing the need for a consistent and general approach in measure theory
			\end{itemize}
		\end{block}
	\end{frame}


\begin{frame}{MEASURE}
	\centering
	\Large WHAT IS MEASURE AND WHY MEASURE?
\end{frame}


\begin{frame}{Why measure?}
	
	In mathematics, we often ask questions like:
	\begin{itemize}
		\item how long is this set?
		\item how large is this region?
		\item what is the chance that an event happens?
	\end{itemize}
	\vspace{0.5cm}
	All these questions involve measure; long (length), large (area), chance (probability), amount (volume), etc. \\ \vspace{0.25cm}
	To answer these questions, we need a systematic way to assign numbers to sets.
	
	
\end{frame}


\begin{frame}{The problem with ``size''}
	
	For simple sets, ``size'' is easy.
	
	\vspace{0.3cm}
	
	Examples:
	\begin{itemize}
		\item the length of an interval $(a,b)$ is $b-a$
		\item the number of elements in a finite set is easy to count
	\end{itemize}
	
	\vspace{0.3cm}
	
	But it is not simple for many sets.
	
	\vspace{0.3cm}
	
	Examples:
	\begin{itemize}
		\item unions of many intervals
		\item infinite sets
		\item complicated events in probability
	\end{itemize}
	
	\vspace{0.3cm}
	
	We need a more general idea of size.
	
\end{frame}


\begin{frame}{Nice Practical Examples}
	
	Now take two intervals that do not overlap.
	
	\vspace{0.3cm}
	
	\[
	A = (1,2), \qquad B = (3,5)
	\]
	
	Their lengths are:
	\begin{itemize}
		\item $\text{length}(A) = 1$
		\item $\text{length}(B) = 2$
	\end{itemize}
	
	\vspace{0.3cm}
	
	The total length of $A \cup B$ is
	\[
	1 + 2 = 3
	\]
	
	So far, everything works nicely.
	
\end{frame}


\begin{frame}{Nice Practical Examples}
	\textbf{Half-open interval example:}
	
	\[
	C = [0,1), \qquad D = [1,3)
	\]
	
	Lengths:
	\begin{itemize}
		\item $\text{length}(C) = 1$
		\item $\text{length}(D) = 2$
	\end{itemize}
	
	The total length of $C \cup D$ is
	\[
	1 + 2 = 3
	\]
	
	Even with half-open intervals, the additivity of length works as expected.
	
\end{frame}


\begin{frame}{Limitations of basic ideas}
	
	Length works well for single intervals,  
	but what about:
	\begin{itemize}
		\item a countable union of intervals?
		\item sets with infinitely many pieces?
		\item random events formed from many outcomes?
	\end{itemize}
	
	Basic formulas are not enough.
	
	\vspace{0.3cm}
	
	So we need a rule that:
	\begin{itemize}
		\item works for simple sets
		\item extends to complicated sets
		\item behaves consistently
	\end{itemize}
	
\end{frame}


\begin{frame}{From size to measure}
	
	A measure is a mathematical tool that:
	\begin{itemize}
		\item assigns a size to sets
		\item works for very general sets
		\item satisfies the natural properties we expect
	\end{itemize}
	
	\vspace{0.3cm}
	
	Length, area, volume, and probability are all special cases of measures.
	
	\vspace{0.3cm}
	
	This motivates the formal definition of a measure.
	
\end{frame}

\begin{frame}{Measure}
	\begin{block}{Definition of Measure}
		Let $\Omega$ be a set and $\mathcal{F}$ a collection of subsets of $\Omega$ (called a \(\sigma\)-algebra).  
		
		A function \(\mu: \mathcal{F} \to [0, \infty]\) is called a \textbf{measure} if it satisfies:
		
		\vspace{0.15cm}
		
		\begin{itemize}
			\item \(\mu(\emptyset) = 0\)
			\item \textbf{Countable additivity:} for any countable collection \(\{A_1, A_2, A_3, \dots\}\) of disjoint sets in \(\mathcal{F}\),
			\[
			\mu\Big(\bigcup_i A_i\Big) = \sum_i \mu(A_i)
			\]
		\end{itemize}
	\end{block}
\end{frame}

\begin{frame}{Measure}
	\begin{block}{Remarks}
		\begin{itemize}
			\item $\mu$ assigns a non-negative extended real number (can be $\infty$) to each measurable set.
			\item The $\sigma$-algebra $\mathcal{F}$ ensures that unions, intersections, and complements of sets are measurable.
			\item Countable additivity is stronger than finite additivity; it works for infinitely many disjoint sets.
		\end{itemize}
	\end{block}
\end{frame}

\begin{frame}{Probability Measure}
	
	A probability measure $P$ is a measure on a $\sigma$-algebra $\mathcal{F}$ of $\Omega$ such that:
	
	\vspace{0.3cm}
	
	\begin{itemize}
		\item $P(A) \ge 0$ for all $A \in \mathcal{F}$
		\item $P(\Omega) = 1$
		\item For any countable collection $\{A_1, A_2, \dots\}$ of disjoint events in $\mathcal{F}$,
		\[
		P\Big(\bigcup_i A_i\Big) = \sum_i P(A_i)
		\]
	\end{itemize}
	
\end{frame}



\begin{frame}{Measure Space}
	\begin{block}{Definition of Measure Space}	
	A measure space is an ordered triple $(\Omega, \mathcal{F}, \mu)$ where:
	
	\vspace{0.3cm}
	
	\begin{itemize}
		\item $\Omega$ is a set, called the sample space.
		\item $\mathcal{F}$ is a $\sigma$-algebra of subsets of $\Omega$, 
		\item $\mu: \mathcal{F} \to [0, \infty]$ is a measure, i.e., a functional satisfying:
		\begin{itemize}
			\item $\mu(\emptyset) = 0$
			\item Countable additivity: for any countable collection $\{A_1, A_2, \dots\}$ of disjoint sets in $\mathcal{F}$,
			\[
			\mu\Big(\bigcup_{i=1}^{\infty} A_i\Big) = \sum_{i=1}^{\infty} \mu(A_i)
			\]
		\end{itemize}
	\end{itemize}
	\end{block}
\end{frame}


\begin{frame}{Measurable Space}
	\begin{block}{Definition of Measurable Space}
		A \textbf{measurable space} is an ordered pair $(\Omega, \mathcal{F})$ where:
		
		\vspace{0.3cm}
		
		\begin{itemize}
			\item $\Omega$ is a set, called the sample space.
			\item $\mathcal{F}$ is a $\sigma$-algebra of subsets of $\Omega$
			
		\end{itemize}
		
		\vspace{0.3cm}
		
	\end{block}
	
\end{frame}



\begin{frame}{Probability Space}
	\begin{block}{Definition of Probability Space}
		A probability space is a triple $(\Omega, \mathcal{F}, P)$ where:
		
		\begin{itemize}
			\item $\Omega$ is the sample space (all possible outcomes)
			\item $\mathcal{F}$ is a $\sigma$-algebra of subsets of $\Omega$ (the events)
			\item $P$ is a probability measure
		\end{itemize}
	\end{block}
	
\end{frame}


\begin{frame}{Properties of a Measure}
	
	Let $(\Omega,\mathcal{F},\mu)$ be a measure space.
	
	\begin{enumerate}
		\item \textbf{Non-negativity:}
		\[
		\mu(A) \ge 0 \quad \text{for all } A \in \mathcal{F}.
		\]
		
		\item \textbf{Null empty set:}
		\[
		\mu(\emptyset) = 0.
		\]
		
		\item \textbf{Countable additivity (sigma-additivity):}  
		If $\{A_n\}_{n=1}^\infty \subseteq \mathcal{F}$ are pairwise disjoint, then
		\[
		\mu\!\left(\bigcup_{n=1}^\infty A_n\right)
		=
		\sum_{n=1}^\infty \mu(A_n).
		\]
		
		\item \textbf{Finite additivity:}  
		If $A,B \in \mathcal{F}$ and $A \cap B = \emptyset$, then
		\[
		\mu(A \cup B) = \mu(A) + \mu(B).
		\]
		
		\item \textbf{Monotonicity:}  
		If $A \subseteq B$, then
		\[
		\mu(A) \le \mu(B).
		\]
	\end{enumerate}
	
\end{frame}


\begin{frame}{Properties of a Measure (cont.)}
	
	\setcounter{enumi}{5}
	
	\begin{enumerate}
		\item \textbf{Continuity from below:}  
		If $A_1 \subseteq A_2 \subseteq \cdots$ and
		\[
		A = \bigcup_{n=1}^\infty A_n,
		\]
		then
		\[
		\mu(A) = \lim_{n\to\infty} \mu(A_n).
		\]
		
		\item \textbf{Continuity from above:}  
		If $A_1 \supseteq A_2 \supseteq \cdots$, $\mu(A_1) < \infty$, and
		\[
		A = \bigcap_{n=1}^\infty A_n,
		\]
		then
		\[
		\mu(A) = \lim_{n\to\infty} \mu(A_n).
		\]
		
		\item \textbf{Inclusion--exclusion (two sets):}
		\[
		\mu(A \cup B) = \mu(A) + \mu(B) - \mu(A \cap B).
		\]
	\end{enumerate}
	
\end{frame}


\begin{frame}{Practical Examples of Measures}
	
	\textbf{Length measure on $\mathbb{R}$:}  $\mu((a,b)) = b - a$\\
	\[
	\text{Example: } \mu([2,5]) = 5-2 = 3
	\]
	
	
	\textbf{Counting measure on any set $\Omega$: } $ \mu(A) = \text{number of elements in set } A$
	\[
	\mu(\{x, y, z\}) = 3
	\]
	
	\textbf{Probability measure on a probability space $(\Omega, \mathcal{F}, P)$:} \\
	Let the event of rolling an even number be, $A = \{2, 4, 6\}.$ \\
	Since each event $A_i$ of $\Omega$ has a probability of $\frac{1}{6}.$ \\
	It is easy to prove that\\
	
	\vspace{0.15cm}
	
	\begin{itemize}
		\item $P(\Omega) = P(\{1\}) + P(\{2\}) + \cdots + P(\{6\}) = 1$
		
		\vspace{0.15cm}
		\item $P(A_1 \cup A_2 \cup A_3) = P(A) = \frac{1}{6} + \frac{1}{6} + \frac{1}{6} = \frac{3}{6} = \frac{1}{2}$
		
	\end{itemize}	
\end{frame}


\begin{frame}{Classical Measures}
	\begin{block}{Lengths in $\mathbb{R}$}
		Length in $\mathbb{R}$: For any interval $I \in \mathbb{R}$ bounded by the endpoints $a,b$ , its length is given as \\ $\ell(I)$ = $|b-a|$.
		\vspace{1cm}
		\begin{exampleblock}{Additivity property of length}
			If an interval $I$ is a disjoint union of a finite family $\{I_k\}_{k=1}^{n}$ of intervals, then  $\ell(I) = \sum_{k=1}^{n} \ell(I_k)$.
		\end{exampleblock}
		
		
	\end{block}
\end{frame}

\begin{frame}{Classical examples of Measure}
	\begin{block}{Areas in $\mathbb{R}^2$}
		Given that $I, J$ are the intervals (or lengths) of any rectangle $A$, then \\ area$(A) = \ell(I)\ell(J)$.
		\vspace{1cm}
		\begin{exampleblock}{Additive property of Area}
			If $A$ is a rectangle of disjoint union of a finite family of rectangles $A_1, A_2, \dots, A_n$, then area$(A) = \sum_{k=1}^{n}$ area$(A_k)$.
		\end{exampleblock}
	\end{block}
\end{frame}

\begin{frame}{Classical examples of Measure}
	\begin{block}{Volumes in $\mathbb{R}^3$}
		Any box in $\mathbb{R}^3$ of the form $A=I$x$J$x$K$, where $I, J, K$ are intervals in $\mathbb{R}$, will yield the set vol$(A) = \ell(I) \ell(J) \ell(K)$.
		\begin{itemize}
			\item The additive property of volume follows similarly.
		\end{itemize}
		
	\end{block}
	\vspace{1cm}
	\begin{block}{Probability}
		If the event $A \subset \Omega$, and $A$ is a disjoint union of a finite sequence of events $A_1, \dots, A_n$, then $\mathbb{P}(A) = \sum_{k=1}^{n} \mathbb{P}(A_k)$.
		
	\end{block}
\end{frame}

\begin{frame}{}
	
	\begin{exampleblock}{Note these similarities}
		All the above had the following.
		\begin{itemize}
			\item A non-empty set $\Omega$ (i.e. $\mathbb{R}, \mathbb{R}^2, \mathbb{R}^3, \Omega)$.
			\item A family of subsets $S$ (i.e. intervals, rectangles, boxes, events).
			\item A functional $\mu: S \to \mathbb{R}_+ :=[0,+\infty)$ (length, area, etc.) with the following property:
			\begin{itemize}
				\item if $A \in S$ is a dsijoint union of a finite family $\{A_k\}_{k=1}^{n}$ of sets from $S$, then $\mu(A) = \sum_{k=1}^{n} \mu(A_k)$.
			\end{itemize}
		\end{itemize}
	\end{exampleblock}
\end{frame}


\begin{frame}{Lebesgue Measures}
	
	\begin{itemize}
		\item \textbf{Length:} Lebesgue measure on $\mathbb{R}$ (1-dimensional)
		\item \textbf{Area:} Lebesgue measure on $\mathbb{R}^2$ (2-dimensional)
		\item \textbf{Volume:} Lebesgue measure on $\mathbb{R}^3$ (3-dimensional)
		\item \textbf{Probability:} Lebesgue measure on a probability space assigns probabilities to events (for continuous spaces)
	\end{itemize}
	
	\vspace{0.5cm}
	All of the above are examples of Lebesgue measures in different contexts. They generalize the concept of “size” — whether length, area, volume, or probability — in a mathematically rigorous way that works even for very irregular sets.
	
\end{frame}


\begin{frame}{$\sigma$-additive measures}
	Let $\Omega$ be a non-empty set and $S$ be a family of subsets of $\Omega$. 
	\begin{block}{Definition of $\sigma$-additive measures}
		A functional $\mu: S \to \mathbb{R}_+$ is called a \textbf{$\sigma$-additive measure} if whenever
		\begin{itemize}
			\item a set $A \in S$ is a disjoint union of an at most countable sequence $\{A_k\}_{k=1}^{N}$ (where $N$ is either finite or $N = \infty$), then
			\item $\mu(A) = \sum_{k=1}^{N} \mu(A_k)$.
		\end{itemize}
	\end{block}
	
	\begin{block}{Remark}
		\begin{itemize}
			\item $\mu$ is a \textit{finitely additive measure} if this property holds for finite values of $N$.
			\item Every $\sigma$-additive measure is finitely additive, but the converse is not true.
		\end{itemize}
	\end{block}
\end{frame}


\begin{frame}{Proof that $\sigma$-additivity $\implies$ finite additivity}	
	Let $\mu$ be a $\sigma$-additive measure on a $\sigma$-algebra $\mathcal{F}$.  
	
	Take any two disjoint sets $A, B \in \mathcal{F}$.  
	
	Consider the countable sequence
	\[
	A_1 = A, \quad A_2 = B, \quad A_3 = \emptyset, \quad A_4 = \emptyset, \dots
	\]
	
	Since the sets are disjoint, $\sigma$-additivity gives:
	\[
	\mu\Big(\bigcup_{n=1}^{\infty} A_n\Big) = \sum_{n=1}^{\infty} \mu(A_n).
	\]
	
	The left-hand side is
	\[
	\mu(A \cup B),
	\]
	and the right-hand side is
	\[
	\mu(A) + \mu(B) + 0 + 0 + \dots = \mu(A) + \mu(B).
	\]
	
	Hence,
	\[
	\mu(A \cup B) = \mu(A) + \mu(B),
	\]
	which is exactly finite additivity.
	
	Therefore, $\sigma$-additivity automatically implies finite additivity.
	
\end{frame}


\begin{frame}{$\sigma$-additive measure}
	\begin{block}{Theorem }
		The length is a $\sigma$-additive measure on the family of all bounded intervals in $\mathbb{R}$. 
	\end{block}
\end{frame}



\begin{frame}{$\sigma$-subadditive measures}
	\begin{block}{Definition}
		A functional $\mu: \mathcal{S} \to \mathbb{R}_+$ is called $\sigma$-subadditive if whenever $A \subset \displaystyle \bigcup_{k=1}^{N} A_k$ where $A$ and $A_k$ are all elements of $\mathcal{S}$ and $N$ is either finite or infinite, \\
		$\mu(A) \leq \displaystyle \sum_{k=1}^{N} \mu(A_k)$.
		
		\begin{alertblock}{Note}
			If this property holds for finite values of $N$, then $\mu$ is called finitely subadditive.
		\end{alertblock}
	\end{block}
\end{frame}

\begin{frame}{Lemma }
	\begin{block}{}
		\begin{center}
			The \textit{length} is $\sigma$-subadditive.
		\end{center}
	\end{block}
\end{frame}

\begin{frame}{Proof}
	Let $I$, $\{I_k\}_{k=1}^{\infty}$ be intervals such that $I \subset \cup_{k=1}^{\infty} I_k$, we want to prove that \\
	$\ell(I) \leq \displaystyle \sum_{k=1}^{\infty} \ell(I_k)$.\\ \vspace{0.15cm}
	Let us fix some $\varepsilon > 0$ and choose a bounded closed interval $I' \subset I$ such that
	$\ell(I) \leq \ell(I')$ + $\varepsilon$.\\ \vspace{0.15cm}
	For any $k$, choose an open interval $I_{k}^{'} \supset I_k$ such that \\ $\ell(I'_k) \leq \ell(I_k)$ + $\frac{\varepsilon}{2^k}$.\\ \vspace{0.15cm}
	Then the bounded closed interval $I'$ is covered by a sequence $\{I'_k\}$ of open intervals. There is a finite subfamily $\{I'_{k_j}\}_{j=1}^{n}$ that also covers $I'$. 
	
\end{frame}

\begin{frame}{Proof}
	It follows from the finite additivity of length that it is finitely subadditive. That is, 
	\[
	\ell(I') \leq \sum_{j} \ell(I'_{k_j}) \implies 
	\ell(I') \leq \sum_{k=1}^{\infty} \ell(I'_k).
	\]
	
	This yields 
	\[
	\ell(I) \leq \varepsilon + \sum_{k=1}^{\infty}\left(\ell(I_k) + \frac{\varepsilon}{2^k}\right) 
	= 2\varepsilon + \sum_{k=1}^{\infty} \ell(I_k).
	\]
	
	Since $\varepsilon > 0$ is arbitrary, letting $\varepsilon \to 0$ finishes the proof. $\square$
\end{frame}


\begin{frame}{Theorem}
	\begin{block}{}
		The length is a $\sigma$-additive measure on the family of all bounded intervals in $\mathbb{R}$.
	\end{block}
\end{frame}

\begin{frame}{Proof}
	We need to prove that if
	$I = \displaystyle \bigsqcup_{k=1}^{\infty} I_k$, then $\ell(I) = \sum_{k=1}^{\infty} \ell(I_k)$.\\
	By the $\sigma$-subadditive lemma, we have $\ell(I) \leq \displaystyle \sum_{k=1}^{\infty} \ell(I_k)$, so we need to prove the opposite inequality. \\
	For a fixed $n \in \mathbb{N}$, we have \\ 
	$I \supset \displaystyle \bigsqcup_{k=1}^{n} I_k$.\\
	It follows from the finite additivity of length that \\
	$\ell(I) \ge \displaystyle \sum_{k=1}^{n} \ell(I_k)$.
\end{frame}

\begin{frame}{Proof}
	Letting $n \to \infty$, we obtain \\
	$\ell(I) \ge \displaystyle \sum_{k=1}^{\infty} \ell(I_k)$\\
	which finishes the proof.
\end{frame}

\end{document}